\chapter*{Список сокращений и условных обозначений} % Заголовок
\addcontentsline{toc}{chapter}{Список сокращений и условных обозначений}  % Добавляем его в оглавление
\noindent
%\begin{longtabu} to \dimexpr \textwidth-5\tabcolsep {r X}
\begin{longtabu} to \textwidth {r X}
% Жирное начертание для математических символов может иметь
% дополнительный смысл, поэтому они приводятся как в тексте
% диссертации

\textbf{ANSI} & American National Standards Institute, Incorporated, Американский национальный институт стандартов\\

\textbf{IEEE} & Institute of Electrical and Electronics Engineers, Институт инженеров электротехники и электроники\\

\textbf{IEC, МЭК} & International Electrotechnical Commission, Международная электротехническая комиссия\\

\textbf{CIRED} & Congres International des Reseaux Electriques de Distribution, Международная конференция по распределению электроэнергии\\

\end{longtabu}

\addtocounter{table}{-1}% Нужно откатить на единицу счетчик номеров таблиц, так как предыдующая таблица сделана для удобства представления информации по ГОСТ

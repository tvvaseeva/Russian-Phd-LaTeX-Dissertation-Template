\chapter{Анализ и контроль показателей качества электрической энергии в электроэнергетических системах}\label{ch:ch1}

В первом разделе рассматривается понятие, характеризующее КЭ в электрических сетях систем электроснабжения переменного тока с частотами 50 и 60~Гц и порядок оценки результатов. Стандарты в области контроля определяющие состав показателей качества электрической энергии, методику измерений и характеристики средств измерений. В результате анализа производится сравнение государственных стандартов Российской Федерации и международных стандартов (МЭК, IEC – International Electrotechnical Commission), которые описывают требования и нормы КЭ \cite{IEC}.
%\hyperlink{https://www.iec.ch/}{[199]}. 
% [199 - International Electrotechnical Commission [Электронный ресурс] – Режим доступа:https://www.iec.ch/]

В первом разделе произведен сравнительный анализ существующих средств контроля учета показателей КЭ. В заключении описывается основные задачи и виды контроля КЭ.

\section{Понятие качества электрической энергии}\label{sec:ch1/sec1_1}
Повышение КЭ является основным фактором улучшения энергетической эффективности промышленных предприятий \cite{закон2009энергосбережении}. 
% [10 - Закон Ф. Об энергосбережении и о повышении энергетической эффективности и о внесении изменений в отдельные законодательные акты Российской Федерации //№ 261-ФЗ от 23.11. – 2009.]
%http://www.consultant.ru/document/cons_doc_LAW_93978/

Понятие качества электрической энергии (КЭ) описано ГОСТ 32144–2013~\cite{ГОСТ32144-2013} п.3.1.38, обозначает степень соответствия характеристик электрической энергии в данной точке электрической системы совокупности нормированных показателей КЭ.

Задача исследования -- улучшить показатели качества электрической энергии (ПКЭ) в электрических сетях систем электроснабжения, совершенствуя средства измерительных устройств для определения гармоник. При разработке технического устройства должны учитываться ПКЭ  согласно действующим нормативам. 

Введен целый ряд нормативных документов (МЭК, ГОСТ). Изменение стандартов в области измерений электрической энергии показано на рисунке \cite{ГОСТ30804.4.30-2013, ГОСТ30804.4.7-2013, ГОСТ32144-2013, ГОСТР8.655-2009, ГОСТР51317.4.15-2012,ГОСТ33073-2014,ГОСТ8.622-2013}. 
 
\begin{figure}[ht]
	\centerfloat{
		\includegraphics[scale=0.7]{picture1}
	}
	\caption{Развитие государственных стандартов в области контроля КЭ.}\label{fig:picture1}
\end{figure}
%Действующие стандарты на рисунке

% [179 - ГОСТ 30804.4.30-2013 (IEC 61000-4-30:2008) Электрическая энергия. Совместимость технических средств электромагнитная. Методы измерений показателей качества электрической энергии (с Поправкой) [Электронный ресурс] / Режим доступа: http://docs.cntd.ru/document/1200104665]

% [4 - ГОСТ 30804.4.7-2013 (IEC 61000-4-7:2009) Совместимость технических средств электромагнитная. Общее руководство по средствам измерений и измерениям гармоник и интергармоник для систем электроснабжения и подключаемых к ним технических средств (с Поправкой) [Электронный ресурс] / Режим доступа: http://docs.cntd.ru/document/1200103652]

% [125 - ГОСТ 32144-2013 Электрическая энергия. Совместимость технических средств электромагнитная. Нормы качества электрической энергии в системах электроснабжения общего назначения [Электронный ресурс] – Режим доступа: http://docs.cntd.ru/document/1200104301/]

% [136 - ГОСТ Р 8.655-2009 Государственная система обеспечения единства измерений (ГСИ). Средства измерений показателей качества электрической энергии. Общие технические требования (с Изменением N 1) [Электронный ресурс] / Режим доступа: http://docs.cntd.ru/document/1200075494/]

% [194 - ГОСТ Р 51317.4.15-2012 (МЭК 61000-4-15:2010) Совместимость технических средств электромагнитная. Фликерметр. Функциональные и конструктивные требования [Электронный ресурс] / Режим доступа: http://docs.cntd.ru/document/1200096463]

% [195 - ГОСТ 33073-2014 Электрическая энергия. Совместимость технических средств электромагнитная. Контроль и мониторинг качества электрической энергии в системах электроснабжения общего назначения (с Поправкой) [Электронный ресурс] / Режим доступа: http://docs.cntd.ru/document/1200115349]

% [196 - ГОСТ 8.622-2013 Государственная система обеспечения единства измерений (ГСИ). Испытательное оборудование для определения коэффициента проникания тест-аэрозоля через средства индивидуальной защиты органов дыхания. Методика аттестации [Электронный ресурс] / Режим доступа: http://docs.cntd.ru/document/1200108162]

Основополагающим нормативным документом в Российской Федерации, регламентирующим положения связанные с КЭ в Российской Федерации, был ГОСТ 13109–97 \cite{ГОСТ13109-97}. 

Стандарт не отвечал современным реалиям и заменен на ГОСТ 32144-2013 \cite{ГОСТ32144-2013}, действует с 01.07.2014 г. приказом Росстандарта. Российский стандарт соответствует европейскому ЕN 50160:2010 «Характеристики напряжения электричества, поставляемого общественными распределительными сетями» \cite{ЕN50160:2010}. 
% [11 - 11.	EN 50160:2010 Voltage characteristics of electricity supplied by public electricity networks [Электронный ресурс] / Режим доступа: https://infostore.saiglobal.com/preview/98699522296.pdf?sku=859794_saig_nsai_nsai_2045468]

Действующий зарубежный стандарт DIN EN 50160-2011 \cite{DINEN50160-2011}.

Новый стандарт ГОСТ 32144–2013 по требованиям к КЭ учитывает рекомендации положений международных стандартов и новых национальных стандартов по методам и средствам измерения и оценки ПКЭ, а также сближает структуру и положения данного стандарта с европейским стандартом ЕN 50160:2010.
Определение понятия средства измерений (СИ) в ГОСТ 30804.4.7-2013 \cite{ГОСТ30804.4.7-2013}.

Некоторые принципиальные отличия  ГОСТ 32144–2013 \cite{ГОСТ32144-2013} от предыдущего стандарта ГОСТ 13109–97 \cite{ГОСТ13109-97}: 
%\noindent Маркированный список:
\begin{itemize}
	\item Отличие по интервалам усреднения показателям качества электроэнергии (отклонение частоты -- 10 сек., вместо 20 сек. в ГОСТ 13109-97; не симметрия напряжения -- интервал усреднения 10 мин., вместо 3 сек. в ГОСТ 13109-97; гармонические составляющие напряжения - 10 мин., вместо 3 сек. в ГОСТ 13109-97) с интервалом периода в одну неделю, вместо суток в ГОСТ 13109-97.
	\item Измерения согласно ГОСТ 30804.4.30-2013 (IEC 61000-4-30:2008) \cite{ГОСТ30804.4.30-2013} и ГОСТ 30804.4.7–2013 (IEC 61000-4-7:2009) \cite{ГОСТ30804.4.7-2013}.
	\item Для медленных изменений напряжения исключены режимы наименьших и наибольших нагрузок.
	\item Добавлены таблицы классификации провалов напряжения по остаточному напряжению.
\end{itemize}

Кроме Российских стандартов, существуют Международные регламентирующие документы по мониторингу КЭ, рекомендация от Института инженеров электротехники (IEEE -- Institute of Electrical and Electronics Engineers) \cite{IEEE_PES}. 
% [12 -	IEEE PES Power Quality Subcommittee [Электронный ресурс] / Режим доступа: https://site.ieee.org/pes-pq/]

IEEE PES Power Quality Subcommittee (IEEE PES по качеству электроэнергии) участвует в качестве представителя Национального комитета США в Международной конференции и выставке по распределению электроэнергии (CIRED -- Congres International des Reseaux Electriques de Distribution) \cite{CIRED, CIRED_CONFERENCE}. 
% [197 - INTERNATIONAL CONFERENCE ON ELECTRICITY DISTRIBUTION  [Электронный ресурс] / Режим доступа: http://www.cired.net/]

%[198 - THE 25TH INTERNATIONAL CONFERENCE AND EXHIBITION ON ELECTRICITY DISTRIBUTION [Электронный ресурс] / Режим доступа: http://www.cired2019.org/page/about-cired]

Группа международных стандартов в области контроля КЭ:
\begin{itemize}
	\item Стандарт IEEE 519-2014 - Рекомендуемая практика и требования IEEE для гармонического управления в электроэнергетических системах (IEEE 519-2014 - IEEE Recommended Practice and Requirements for Harmonic Control in Electric Power Systems)\cite{IEEE_519-2014}. 
	\item Стандарт IEEE 1159-2019: рекомендуемая практика IEEE для мониторинга качества электроэнергии (1159-2019 - IEEE Recommended Practice for Monitoring Electric Power Quality) \cite{IEEE_1159-2019}. 
	\item Стандарт 1159.3-2019: рекомендуемая практика IEEE для формата обмена данными о качестве электроэнергии (1159.3-2019 - IEEE Recommended Practice for Power Quality Data Interchange Format (PQDIF)\cite{IEEE_1159.3-2019}.
	\item Стандарт IEEE 1250-2018 - Руководство IEEE по выявлению и улучшению качества напряжения в энергосистемах (IEEE 1250-2018 - IEEE Guide for Identifying and Improving Voltage Quality in Power Systems)\cite{IEEE_1250-2018}.
	\item Стандарт IEEE 1409-2012 - Руководство IEEE по применению силовой электроники для повышения качества электроэнергии в распределительных системах. Номинальные значения: от 1 до 38~кВ. (IEEE 1409-2012 - IEEE Guide for Application of Power Electronics for Power Quality Improvement on Distribution Systems Rated 1 kV Through 38 kV)\cite{IEEE_1409-2012}.
	\item Стандарт IEEE 1453-2015 - рекомендуемая практика IEEE для анализа колеблющихся установок в энергосистемах (IEEE 1453-2015 - IEEE Recommended Practice for the Analysis of Fluctuating Installations on Power Systems).
	\item Стандарт IEEE 1564-2014 - Руководство IEEE по показателям падения напряжения (IEEE 1564-2014 - IEEE Guide for Voltage Sag Indices) \cite{IEEE_1453-2015}.
\end{itemize}

На рисунке \ref{img:picture2} изображены действующие международные стандарты IEEE. 
\begin{figure}[ht]
	\centering
	\includegraphics [scale=0.9] {picture2}
	\caption{Группа международных стандартов в области контроля КЭ}
	\label{img:picture2}
\end{figure}

Группа международных стандартов описывает факторы, влияющие на КЭ \cite{IEEE_519-2014, IEEE_1159-2019, IEEE_1159.3-2019, IEEE_1250-2018, IEEE_1564-2014, IEEE_1409-2012, IEEE_1453-2015, IEEE_1159-2009}:

\begin{itemize}
	\item IEC 61000-4-30:~2015 - Электромагнитная совместимость. Часть 4-30. Методы испытаний и измерений. Методы измерения качества электроэнергии \cite{IEC61000-4-30:2015}.
	\item IEC TR 61000-1-1:~1992 - Электромагнитная совместимость. Часть 1. Общие положения. Раздел 1. Применение и толкование основных определений и терминов \cite{IEC_TR_61000-1-1:1992}.
	\item IEC 61000-1-2:~2016 - Электромагнитная совместимость. Часть 1-2. Общие положения. Методология обеспечения функциональной безопасности электрических и электронных систем, включая оборудование в отношении электромагнитных явлений \cite{IEC61000-1-2:2016}.
	\item IEC TR 61000-1-3:~2002 - Электромагнитная совместимость (ЭМС). Часть 1-3. Общие положения. Воздействие высотной ЭМИ (HEMP) на гражданское оборудование и системы \cite{IECTR61000-1-3:2002}.
\end{itemize}

\begin{figure}[ht]
	\centering
	\includegraphics [scale=0.9] {picture3}
	\caption{Стандарты, в области контроля КЭ}
	\label{img:picture3}
\end{figure}
%[200 - IEC 61000-4-30:2015 Electromagnetic compatibility (EMC) - Part 4-30: Testing and measurement techniques - Power quality measurement methods [Электронный ресурс] / Режим доступа: https://webstore.iec.ch/publication/21844#additionalinfo]

%[201 - IEC TR 61000-1-1:1992 Electromagnetic compatibility (EMC) - Part 1: General - Section 1: Application and interpretation of fundamental definitions and terms [Электронный ресурс] / Режим доступа: https://webstore.iec.ch/publication/4120] 

%[202 - IEC 61000-1-2:2016 Electromagnetic compatibility (EMC) - Part 1-2: General - Methodology for the achievement of functional safety of electrical and electronic systems including equipment with regard to electromagnetic phenomena [Электронный ресурс] / Режим доступа: https://webstore.iec.ch/publication/24517]

%[203 - IEC TR 61000-1-3:2002 Electromagnetic compatibility (EMC) - Part 1-3: General - The effects of high-altitude EMP (HEMP) on civil equipment and systems [Электронный ресурс] / Режим доступа: https://webstore.iec.ch/publication/4122]

Национальный комитет США Международной электротехнической комиссии (USNC~/~IEC) предоставляет стратегию для эффективного участия в разработке стандартов. USNC участвует почти во всей технической программе Международной электротехнической комиссии (МЭК) и управляет многими ключевыми комитетами и подгруппами.

IEC является ведущей глобальной организацией, которая готовит и публикует международные стандарты для электрических, электронных и смежных технологий. Они служат основой для национальной стандартизации и справочными материалами при разработке международных тендеров и контрактов.
USNC/IEC является полностью интегрированным комитетом Американского национального института стандартов (ANSI – American National Standards Institute, Incorporated). ANSI служит ключевым ресурсом для стандартов и информации \cite{ANSI}.
%[204 - American National Standards Institute [Электронный ресурс] / Режим доступа: https://www.ansi.org/standards_activities/iec_programs/overview]

Два российских стандарта были разработаны на основе стандартов Международной Электротехнической Комиссии (International Electrotechnical Commission) IEC 61000-4-30:2008 и IEC 61000-4-7-2002:
\begin{itemize}
	\item ГОСТ 30804.4.30–2013 (IEC 61000-4-30:2011) Методы измерений показателей качества электрической энергии \cite{ГОСТ30804.4.30-2013}.
	\item ГОСТ 30804.4.7–2013 (IEC 61000-4-7:2011) Общее руководство по средствам измерений и измерениям гармоник и интергармоник для систем электроснабжения и подключаемых к ним технических средств \cite{ГОСТ30804.4.7-2013}.
\end{itemize}

\begin{figure}[p]
	\centering
	\includegraphics [scale=0.9] {picture4}
	\caption{Стандарты, описывающие измерения параметров сети}
	\label{img:picture4}
\end{figure}

\section{Показатели качества электроэнергии, характеризующие несинусоидальность напряжения} \label{sec:ch1/sec1_2}

К проблемам КЭ относится множество явлений. Различные причины способствуют на улучшение КЭ и характеристик оборудования \cite{ГОСТ13109-97}.

\begin{table} [p]%
	\caption{Свойства ЭЭ, ПКЭ, виновники ухудшения КЭ}%
	\label{tbl:test3}% label всегда желательно идти после caption
	\begin{SingleSpace}
		\setlength\extrarowheight{6pt} %вот этим управляем расстоянием между рядами, \arraystretch даёт неудачный результат
		\setlength{\tymin}{1.9cm}% минимальная ширина столбца
		\begin{tabulary}{\textwidth}{@{}>{\zz}L >{\zz}C >{\zz}C @{}}% Вертикальные полосы не используются принципиально, как и лишние горизонтальные (допускается по ГОСТ 2.105 пункт 4.4.5) % @{} позволяет прижиматься к краям
			\toprule     %%% верхняя линейка
			Показатель КЭ & 
			Свойства ЭЭ &
			Виновники ухудшения КЭ \\
			\midrule %%% тонкий разделитель. Отделяет названия столбцов. Обязателен по ГОСТ 2.105 пункт 4.4.5 
			Установившееся отклонение напряжения ${\delta U_y}$ &
			Отклонение напряжения &
			Энергоснабжающая организация \\
			
			Размах изменения напряжения~${\delta U_t}$ 
			
			Доза фликера ${P_t}$ &
			Колебания напряжения &
			Потребитель с переменной нагрузкой \\
			
			Коэффициент искажения синусоидальности кривой напряжения ${K_U}$ &
			Несинусоидальность напряжения &
			Потребитель с нелинейной нагрузкой \\
			
			Коэффициент $n$-oй гармонической составляющей напряжения ${K_{U(n)}}$ &
			Несинусоидальность напряжения &
			Потребитель с нелинейной нагрузкой \\
			
			Коэффициент несимметрии напряжений по обратной последовательности ${K_{2U}}$ &
			Несимметрия трехфазной системы напряжений &
			Потребитель с несимметричной нагрузкой \\
			
			Коэффициент несимметрии напряжений по нулевой последовательности ${K_{0U}}$ &
			Несимметрия трехфазной системы напряжений &
			Потребитель с несимметричной нагрузкой \\
			
			Отклонение частоты ${\Delta f}$ &
			Отклонение частоты &
			Энергоснабжающая организация \\
			
			Длительность провала напряжения ${\Delta t_p}$ & % Нужно вставить русский символ в формулу ∆t_п
			Провал напряжения &
			Энергоснабжающая организация \\
			
			Импульсное напряжение ${U_{imp}}$ &
			Импульс напряжения &
			Энергоснабжающая организация \\
			
			Коэффициент временного перенапряжения ${K_{per U}}$ &
			Временное перенапряжение &
			Энергоснабжающая организация \\
			
			\bottomrule %%% нижняя линейка
		\end{tabulary}%
	\end{SingleSpace}
\end{table}

Отклонение напряжение – это отличие фактического напряжения работы систем электроснабжения от его номинального значения. Отклонение напряжения происходит под воздействием медленного изменения нагрузки в той или иной точке сети. Это отрицательно влияет на качество и срок службы бытовой электроприборов. Отклонение напряжение влияет на работу электросварочных машин. Так для точечной сварки отклонение напряжения на 15 \% приводит к браку продукции. Чрезмерное повышение напряжение приводит к росту нагрузки тока и мощности короткого замыкания. Для приборов с электрическими схемами реальную опасность представляет перегрев, сбой элементов схемы управления. Если элементы находятся, включены длительное время, то может произойти выход из строя элементов и прибор перестанет выполнять заложенные в него функции.

В диссиртационной работе рассматривается оценка ПКЭ, характеризующая несинусоидальность напряжения. В ГОСТ 32144-2013 \cite{ГОСТ32144-2013}, п. 4.2.4 Несинусоидальность напряжения – искажение синусоидальной формы кривой напряжения.

Гармонические составляющие напряжения обусловлены нелинейными нагрузками электрических сетей, которые подключаются к сетям различного напряжения. Гармонические токи, полные сопротивления электрических сетей, напряжения гармонических составляющих в точках передачи ЭЭ изменяются во времени.

Если нагрузка в системе линейная, то и токи во всех ветвях синусоидальны. Наличие нелинейной нагрузки приводит к возникновению несинусоидальных токов во всех ветвях электрической сети, что приводит к возникновению несинусоидальной кривой напряжения во всех точках сети, что отрицательно влияет на работу электрической сети. 
% Нужен ли рисунок Несинусоидальность напряжения?

Нелинейные и коммутируемые нагрузки могут вызвать искажения нормальных синусоидальных сигналов тока и напряжения в системе переменного тока. Источники искажений: силовые трансформаторы, преобразовательные устройства переменного тока.

Анализ несинусоидальности напряжения является составной частью системы эксплуатационного контроля КЭ. Для этого на шинах управления соответствующих контрольных пунктов устанавливают анализаторы несинусоидальности, которые соединены с регистрирующими приборами. Разлагают на спектральные составляющие, чтобы проанализировать несинусоидальность режимов напряжения. Поэтому, для повышения точности измерения показателей несинусоидальности напряжения необходимо повышать точность оценки гармонических и интергармонических составляющих напряжения. 

Существуют различия между утратившим силу стандартом ГОСТ 13109–97 \cite{ГОСТ13109-97} и действующим ГОСТ 32144-2013 \cite{ГОСТ32144-2013}.

Сравнительный анализ стандартов качества электрической энергии ГОСТ 13109–97 и ГОСТ 32144–2013 \cite[с.~155]{Киселёв}
% [205 - Киселёв Б. Ю. Сравнительный анализ стандартов качества электрической энергии ГОСТ 13109–97 и ГОСТ 32144–2013 // Молодой ученый. — 2016. — №20. — С. 155-157. — URL https://moluch.ru/archive/124/34114/ (дата обращения: 09.10.2019)]


Основные отличия в ГОСТ 32144-2013:
\begin{itemize}
	\item Изменен интервал времени, соответствующий расчетному интервалу времени на одну неделю.
	\item Изменения характеристик ЭЭ разделены на две категории – продолжительные изменения характеристик напряжения и случайные события.
	\item Процедура проведения контроля производится на основе ГОСТ Р 51317.4.30–2008 \cite{ГОСТР51317.4.30-2008} и ГОСТ Р 51317.4.7–2008 \cite{ГОСТР51317.4.7-2008}.
	\item Введены интергармонические составляющие напряжения, хотя ни каких ограничений по их отклонению пока нет, они находятся на стадии разработки.
	\item Вместо коэффициента искажения синусоидальности кривой напряжения несинусоидальность напряжения характеризуется суммарным коэффициентом гармонических составляющих.
\end{itemize} 

% ГОСТ Р 51317.4.30-2008 (Недействующий) -> ГОСТ 30804.4.30-2013 (Действующий)
% [3 - ГОСТ Р 51317.4.30-2008 (МЭК 61000-4-30:2008). Совместимость технических средств электромагнитная. Методы измерений показателей качества электрической энергии [Электронный ресурс] – Режим доступа: http://docs.cntd.ru/document/1200072576]
% [179 - ГОСТ 30804.4.30–2013 (IEC 61000-4-30:2008) Электрическая энергия. Совместимость технических средств электромагнитная. Методы измерений показателей качества электрической энергии [Электронный ресурс] – Режим доступа: http://docs.cntd.ru/document/1200104665]

% ГОСТ Р 51317.4.7–2008 (Недействующий) -> ГОСТ 30804.4.7-2013 (Действующий)
% [4 - ГОСТ 30804.4.7-2013 (IEC 61000-4-7:2009) Совместимость технических средств электромагнитная. Общее руководство по средствам измерений и измерениям гармоник и интергармоник для систем электроснабжения и подключаемых к ним технических средств.]

В ГОСТ 13109-97 \cite{ГОСТ13109-97} коэффициент искажения синусоидальности кривой напряжения $K_U$ определен как \eqref{eq:$K_U$}:

\begin{equation}
\label{eq:$K_U$}
K_U = \frac{\sqrt{\sum_{n=2}^N {U_{(n)}}^2}}{U_{(1)}}\cdot 100 \%
\end{equation}

где $U_{(n)}$ -- Действующее значение $n$-ой гармонической составляющей напряжения;

$n$ -- Порядок гармонической составляющей напряжения;

$N$ -- Порядок последней из учитываемых гармонических составляющих напряжения (стандартом устанавливается $N=40$);

$U_{(1)}$ -- Действующее значение напряжения основной частоты.

\begin{equation}
\label{eq:$K_U2$}
K_U = \frac{\sqrt{\sum_{n=2}^N {U_{(n)}}^2}}{U_{nom}}\cdot 100 \%
\end{equation}

% Таблица 1.2
\begin{table} [p]%
	\caption{Значения коэффициентов нечетных гармонических составляющих напряжения не кратных трем}%
	\label{tbl:test3}% label всегда желательно идти после caption
	\begin{SingleSpace}
		\setlength\extrarowheight{6pt} %вот этим управляем расстоянием между рядами, \arraystretch даёт неудачный результат
		\setlength{\tymin}{1.9cm}% минимальная ширина столбца
		\begin{tabulary}{\textwidth}{@{}>{\zz}C >{\zz}C >{\zz}C >{\zz}C >{\zz}C @{}}% Вертикальные полосы не используются принципиально, как и лишние горизонтальные (допускается по ГОСТ 2.105 пункт 4.4.5) % @{} позволяет прижиматься к краям
			\toprule     %%% верхняя линейка
			Порядок гармонической составляющей~$n$ & 
			\multicolumn{4}{|l|}{Значения коэффициентов гармонических}  \\
			&
			\multicolumn{4}{|l|}{составляющих напряжения $K_{U(n)}$ ,\% ${U_1}$}\\		
			&
			\multicolumn{4}{|l|}{Напряжение электрической сети, кВ} \\			
			&
			$0,38$ кВ &
			$6-25$~кВ &
			$35$ кВ  &
			$110-220$~кВ \\
			\midrule %%% тонкий разделитель. Отделяет названия столбцов. Обязателен по ГОСТ 2.105 пункт 4.4.5 
			$5$ &
			$6$ &
			$4$ &
			$3$ &
			$1,5$ \\
			
			$7$ &
			$5$ &
			$3$ &
			$2,5$&
			$1$ \\
			
			$11$ &
			$3,5$ &
			$2$ &
			$2$ &
			$1$ \\
			
			$13$ &
			$3,0$ &
			$2$ &
			$1,5$ &
			$0,7$\\
			
			$17$ &
			$2,0$ &
			$1,5$ &
			$1$ &
			$0,5$\\
			
			$19$ &
			$1,5$ &
			$1$	&
			$1$ &
			$0,4$\\
			
			$23$ &
			$1,5$ &
			$1$ &
			$1$ &
			$0,4$\\
			
			$25$ &
			$1,5$ &
			$1$ &
			$1$ &
			$0,4$\\
			
			$>25$ &
			$1,5$ &
			$1$ &
			$1$ &
			$0,4$\\
			
			\bottomrule %%% нижняя линейка
		\end{tabulary}%
	\end{SingleSpace}
\end{table}

%Таблица 1.3
\begin{table} [p]%
	\caption{Значения коэффициентов нечетных гармонических  составляющих напряжения кратных трем.}%
	\label{tbl:test3}% label всегда желательно идти после caption
	\begin{SingleSpace}
		\setlength\extrarowheight{6pt} %вот этим управляем расстоянием между рядами, \arraystretch даёт неудачный результат
		\setlength{\tymin}{1.9cm}% минимальная ширина столбца
		\begin{tabulary}{\textwidth}{@{}>{\zz}C >{\zz}C >{\zz}C >{\zz}C >{\zz}C @{}}% Вертикальные полосы не используются принципиально, как и лишние горизонтальные (допускается по ГОСТ 2.105 пункт 4.4.5) % @{} позволяет прижиматься к краям
			\toprule     %%% верхняя линейка
			Порядок гармонической составляющей~$n$ & 
			\multicolumn{4}{|l|}{Значения коэффициентов гармонических}  \\
			&
			\multicolumn{4}{|l|}{составляющих напряжения $K_{U(n)}$ ,\% ${U_1}$}\\		
			&
			\multicolumn{4}{|l|}{Напряжение электрической сети, кВ} \\			
			&
			$0,38$ кВ &
			$6-25$~кВ &
			$35$ кВ  &
			$110-220$~кВ \\
			\midrule %%% тонкий разделитель. Отделяет названия столбцов. Обязателен по ГОСТ 2.105 пункт 4.4.5 
			$3$ &
			$5$ &
			$3$ &
			$3$ &
			$1,5$ \\
			
			$9$ &
			$1,5$ &
			$1$ &
			$1$&
			$0,4$ \\
			
			$15$ &
			$0,3$ &
			$0,3$ &
			$0,3$ &
			$0,2$ \\
			
			$21$ &
			$0,2$ &
			$0,2$ &
			$0,2$ &
			$0,2$\\
			
			
			$>21$ &
			$0,2$ &
			$0,2$ &
			$0,2$ &
			$0,2$\\
			
			\bottomrule %%% нижняя линейка
		\end{tabulary}%
	\end{SingleSpace}
\end{table}

% Таблица 1.4
\begin{table} [p]%
	\caption{Значения нечетных гармонических составляющих напряжения.}%
	\label{tbl:test3}% label всегда желательно идти после caption
	\begin{SingleSpace}
		\setlength\extrarowheight{6pt} %вот этим управляем расстоянием между рядами, \arraystretch даёт неудачный результат
		\setlength{\tymin}{1.9cm}% минимальная ширина столбца
		\begin{tabulary}{\textwidth}{@{}>{\zz}C >{\zz}C >{\zz}C >{\zz}C >{\zz}C @{}}% Вертикальные полосы не используются принципиально, как и лишние горизонтальные (допускается по ГОСТ 2.105 пункт 4.4.5) % @{} позволяет прижиматься к краям
			\toprule     %%% верхняя линейка
			Порядок гармонической составляющей~$n$ & 
			\multicolumn{4}{|l|}{Значения коэффициентов гармонических}  \\
			&
			\multicolumn{4}{|l|}{составляющих напряжения $K_{U(n)}$ ,\% ${U_1}$}\\		
			&
			\multicolumn{4}{|l|}{Напряжение электрической сети, кВ} \\			
			&
			$0,38$ кВ &
			$6-25$~кВ &
			$35$ кВ  &
			$110-220$~кВ \\
			\midrule %%% тонкий разделитель. Отделяет названия столбцов. Обязателен по ГОСТ 2.105 пункт 4.4.5 
			$2$ &
			$2$ &
			$1,5$ &
			$1$ &
			$0,5$ \\
			
			$4$ &
			$1$ &
			$0,7$ &
			$0,5$&
			$0,3$ \\
			
			$6$ &
			$0,5$ &
			$0,3$ &
			$0,3$ &
			$0,2$ \\
			
			$8$ &
			$0,5$ &
			$0,3$ &
			$0,3$ &
			$0,2$ \\
			
			$10$ &
			$0,5$ &
			$0,3$ &
			$0,3$ &
			$0,2$ \\
			
			$12$ &
			$0,2$ &
			$0,2$ &
			$0,2$ &
			$0,2$ \\
			
			$>12$ &
			$0,2$ &
			$0,2$ &
			$0,2$ &
			$0,2$\\
			
			\bottomrule %%% нижняя линейка
		\end{tabulary}%
	\end{SingleSpace}
\end{table}

%Таблица 1.5
\begin{table} [p]%
	\caption{Значения суммарных коэффициентов гармонических составляющих напряжения.}%
	\label{tbl:test3}% label всегда желательно идти после caption
	\begin{SingleSpace}
		\setlength\extrarowheight{6pt} %вот этим управляем расстоянием между рядами, \arraystretch даёт неудачный результат
		\setlength{\tymin}{1.9cm}% минимальная ширина столбца
		\begin{tabulary}{\textwidth}{@{} >{\zz}C >{\zz}C >{\zz}C >{\zz}C @{}}% Вертикальные полосы не используются принципиально, как и лишние горизонтальные (допускается по ГОСТ 2.105 пункт 4.4.5) % @{} позволяет прижиматься к краям
			\toprule     %%% верхняя линейка%			\toprule     %%% верхняя линейка
			\multicolumn{4}{c}{Значения коэффициентов гармонических составляющих напряжения $K_{U(n)}$ ,\% ${U_1}$}  \\
			\multicolumn{4}{c}{Напряжение электрической сети, кВ} \\			
			
			\midrule %%% тонкий разделитель. Отделяет названия столбцов. Обязателен по ГОСТ 2.105 пункт 4.4.5 
			
			$0,38$ кВ &
			$6-25$~кВ &
			$35$ кВ  &
			$110-220$~кВ \\
			%			\midrule %%% тонкий разделитель. Отделяет названия столбцов. Обязателен по ГОСТ 2.105 пункт 4.4.5 				
			\bottomrule %%% нижняя линейка
		\end{tabulary}%
	\end{SingleSpace}
\end{table}

\begin{figure}[p]
	\centering
	\includegraphics [scale=0.9] {picture6}
	\caption{Общая структура СИ}
	\label{img:picture6}
\end{figure}

\section{Обзор приборов контроля показателей КЭ} \label{sec:ch1/sec1_3} 
Показатель качества электрической энергии (ПКЭ) -- это величина, которая характеризует КЭ по одному или нескольким показателям. Определение термина в \cite{ГОСТ33073-2014}, раздел 3.4.
Контроль ПКЭ осуществляется в соответствии с нормативными документами: ГОСТ~32144-2013 \cite{ГОСТ32144-2013},  ГОСТ~33073-2014 \cite{ГОСТ33073-2014}, ГОСТ~30804.4.30-2013 \cite{ГОСТ30804.4.30-2013}. Стандарты установлены при измерении ПКЭ в электрических сетях систем электроснабжения:

\begin{itemize}
	\item Общего назначения однофазного и трехфазного переменного тока с частотой 50 Гц, присоединенных к Единой энергетической системе.
	\item Изолированных систем электроснабжения общего назначения.
	\item Систем электроснабжения промышленных предприятий.
\end{itemize}

Важные характеристики для мониторинга КЭ в отношении параметров электрической энергии: положительное и отрицательное отклонения напряжения, отклонение частоты, суммарный коэффициент гармонических составляющих напряжения, коэффициент -й гармонической составляющей напряжения, коэффициент не симметрии напряжений по обратной последовательности, коэффициент не симметрии напряжений по нулевой последовательности, кратковременная и длительная дозы фликера.
Согласно стандартам, установлены требования к характеристикам средств измерений (СИ) ПКЭ с помощью сертифицированных приборов. Приборы внесены в Государственный реестр СИ в разделе «Сведения об утвержденных типах средств измерений» по приказу Федерального агентства по метрологии и техническому регулированию от 20.08.2014 г. №1286.

К такому роду приборов относятся показывающие и регистрирующие частотомеры и вольтметры, анализаторы качества напряжения, анализаторы несинусоидальности, осциллографы, анализаторы несимметрии, регистраторы искажения формы кривой, электроанализаторы \cite{левин1975качество}.
%[60 - Левин М. С., Мурадян А. Е., Сырых Н. Н. Качество электроэнергии в сетях сельских районов. – Энергия, 1975.]

Средств измерений, типы которых утверждены Росстандартом: вольтметры, амперметры, системы автоматизированные измерительные, счетчики электрической энергии, комплексы аппаратно-программные, преобразователи измерительные и другие.
В настоящее время на рынке СИ ПКЭ представлено большое количество приборов российского и зарубежного производства. Зарубежные приборы удобны и надежны в эксплуатации, но они дороже отечественных аналогов и имеют непродолжительный срок автономной работы.
В работе рассмотрены следующие отечественные СИ:
\begin{enumerate}
	\item <<АКИП-4204/1>>, <<АКИП-4204/2>> (АО <<ПриСТ>>, г.~Москва).
	\item <<ПАРМА РК1.01>>, <<ПАРМА Т400 S>> (ООО <<Парма>>, г.~Санкт-Петербург).
	\item <<ПКЭ-А>> (<<НПП Марс-Энерго>>, г.~Санкт-Петербург).
	\item <<Ресурс-UF2M-3Т52-5-100-1000>> (НПП <<Электротехника>>, г.~Пенза).
	\item <<НЕВА-ПА>> (НПФ <<Энергосоюз>>, г.~Санкт-Петербург).
	\item <<ЩМК96>> (ОАО <<Электроприбор>>, г.~Чебоксары).
	\item <<Прорыв-Т-А>> (НПП <<Прорыв>>, г.~Петрозаводск).	
\end{enumerate}

Торговая марка АКИП (Акционерное общество <<Приборы, Сервис, Торговля>>) изготавливает различные приборы\cite{prist}: 

%[207.	АО «ПриСТ» [Электронный ресурс]. 2019. Режим доступа: https://prist.ru/about/]
\begin{itemize}
	\item Для измерения параметров электробезопасности.
	\item Для анализа качества электроэнергии.
	\item Анализаторы спектра.
	\item Измерители RLC (R -- сопротивление, L -- индуктивность, C -- емкость).
	\item Генераторы импульсов и другие приборы.
\end{itemize}

Компания заказывает у различных производителей Европы, Азии и Америки изготовление средств измерения. Изделия проходят анализ соответствия с требованием стандарта ISO-9000, а также российскими стандартами. Средства измерения проходят на соответствие с ГОСТ-Р. 

Анализаторы спектра фирмы АКИП предназначены для измерения амплитудно-частотных характеристик спектра радиотехнических сигналов. Серия анализаторов 4204 состоит из трех модификаций – 4204, 4204/1, 4204/2. Они отличаются верхней границей диапазона частот. Конструктивно анализаторы выполнены в виде настольного моноблока, объединяющие в своем составе высокочастотные и низкочастотные части и управляющий микропроцессор. Анализаторы работают под управлением встроенного микропроцессора. Они обеспечивают проведение автоматических измерений частотных и амплитудных параметров спектра сигналов. Полученные на приборах спектрограммы могут быть записаны в различных форматах во внутреннюю память, на внешний носитель, а также переданы на компьютер через интерфейс. 


\begin{table} [p]%
	\caption{Сравнительный анализ анализаторов спектра фирмы АКИП.}%
	\label{tbl:test3}% label всегда желательно идти после caption
	\begin{SingleSpace}
		\setlength\extrarowheight{6pt} %вот этим управляем расстоянием между рядами, \arraystretch даёт неудачный результат
		\setlength{\tymin}{1.9cm}% минимальная ширина столбца
		\begin{tabulary}{\textwidth}{@{}>{\zz}C >{\zz}C >{\zz}C >{\zz}C >{\zz}C >{\zz}C @{}}% Вертикальные полосы не используются принципиально, как и лишние горизонтальные (допускается по ГОСТ 2.105 пункт 4.4.5) % @{} позволяет прижиматься к краям
			\toprule     %%% верхняя линейка
			Параметры & 
			
			\multicolumn{5}{|c|}{Торговая марка АКИП}  \\
			
			&
			$4205/2$ &
			$4204/2$ &
			$4204/1$ &
			$4204/1~TG$ &
			$4205/1~TG$ \\
			
			\midrule %%% тонкий разделитель. Отделяет названия столбцов. Обязателен по ГОСТ 2.105 пункт 4.4.5 
			
			Госреестр &
			Да &
			Да &
			Да &
			Да &
			Да \\
			
			Частотный диапазон &
			$9$~кГц - $3,2$~ГГц & 
			$9$~кГц - $7,5$~ГГц &
			$9$~кГц - $1,5$~ГГц &
			$9$~кГц - $1,5$~ГГц &
			$9$~кГц - $2,1$~ГГц \\
			
			Полоса пропускания (RBW) &
			$10$~Гц - $3$~МГц &
			$1$~Гц - $3$~МГц &
			$1$~Гц - $3$~МГц &
			$1$~Гц - $3$~МГц &
			$10$~Гц - $3$~МГц \\
			
			Полоса обзора &	
			$100$~Гц - $3,2$~ГГц &
			$100$~Гц - $3$~ГГц &
			$100$~Гц - $3$~ГГц &
			$100$~Гц - $1,5$ ГГц &
			$100$~Гц - $2,1$~ГГц \\
			
			Гармонические искажения &
			$-65$~дБн &
			$-70$~дБн &
			$-70$~дБн &
			$-70$~дБн &
			$-65$~дБн \\
			
			Уровень собственных шумов &
			$-146$~дБм &
			$-148$~дБм &
			$-148$~дБм &
			$-148$~дБм &
			$-146$~дБм \\
			
			Фазовый шум &
			$-115$~дБн/Гц &
			$-95$~дБн/Гц &
			$-100$~дБн/Гц &
			$-100$~дБн/Гц &
			$-115$~дБн/Гц \\
			
			Максимальный измеряемый уровень &	$+20$~дБм &
			$+30$~дБ &
			$+30$~дБ &
			$+30$~дБ &
			$+20$~дБм \\
			
			
			\bottomrule %%% нижняя линейка
		\end{tabulary}%
	\end{SingleSpace}
\end{table}

Компания <<Парма>> (ООО <<Парма>>, г.~Санкт-Петербург) занимается производством оборудования и систем для электроэнергетики: измерительные приборы, оборудование релейной защиты и автоматики, системы мониторинга переходных режимов, цифровые регистраторы аварийных процессов и другие приборы \cite{parma}. 
%[54 - Компания «ПАРМА» [Электронный ресурс]. 2018. Режим доступа: https://parma.spb.ru/company/about-company/.]

\begin{itemize}
	\item Регистратор качества электрической энергии ПАРМА РК1.01 \cite{parma2} в соответствии с требованиями ГОСТ 32144–2013 \cite{ГОСТ32144-2013}. 
	% Сайт прибора [https://parma.spb.ru/oborudovanie/registratory-kachestva-elektroenergii/]
	Прибор малогабаритный, переносной для регистрации режимов однофазной сети $220$~В. Соответствии с классом S по ГОСТ 30804.4.30–2013 (IEC 61000–4–30:2008) \cite{ГОСТ30804.4.30-2013}. Прибор используется для контроля качества электрической энергии, регистрации графиков нагрузок, а также расследование причин некорректной работы оборудования.
	 
	\item Измерительный преобразователь ПАРМА Т400 S \cite{parma1}.
	% Сайт прибора [https://parma.spb.ru/oborudovanie/mnogofunktsionalnye-izmeritelnye-preobrazovateli/]
	Прибор предназначен для измерения параметров электрической энергии в сетях трехфазного и однофазного тока, а также преобразование информации в цифровой код. Передача данных происходит на микроконтроллер через последовательный интерфейс RS-485. Устройство нижнего уровня в Автоматизированных Информационно-измерительных Системах (АИИС) на объект генерации, передачи и распределения электроэнергии.
\end{itemize}

Компания Марс-Энерго (НПП <<Марс-Энерго>>, г.~Санкт-Петербург) изготавливает измерительные приборы, которые применяются на производственных предприятиях, в различных сферах электроэнергетики и органах Росстандарта. Компания существует в индустрии энергетики более 25~лет \cite{Марс-Энерго}.
%[71 - «НПП Марс-Энерго» [Электронный ресурс]. 1999-2019. Режим доступа: http://www.mars-energo.ru/.]

Энерготестер ПКЭ-А прибор контроля качества электрической энергии \cite{энерготестер}.
% Ссылка на прибор [http://www.mars-energo.ru/home/pribory-kontrolya-kachestva-i-ucheta-elektroenergii/energotester-pke-a.html]
% Руководство эксплуатации [http://www.mars-energo.ru/home/pribory-kontrolya-kachestva-i-ucheta-elektroenergii/energotester-pke-a.html]
Позволяет производить измерения в электросетях трех типов: трехфазной четырехпроводной, трехфазной трехпроводной и однофазной двухпроводной. Измерение и регистрация основных ПКЭ, установленных ГОСТ 32144–2013 \cite{ГОСТ32144-2013} (ГОСТ Р 54149–2010) с оформлением протоколов по ГОСТ 33073–2014 \cite{ГОСТ33073-2014}, в соответствии с ГОСТ 30804.4.30–2013 \cite{ГОСТ30804.4.30-2013} (ГОСТ Р 51317.4.30–2008), ГОСТ 30804.4.7–2013 \cite{ГОСТ30804.4.7-2013} (ГОСТ Р 51317.4.7–2008). 

Устройство позволяет проверять работоспособность и правильность подключения энергетических измерительных преобразователей напряжения, тока, активной и реактивной мощности на местах их эксплуатации. Измерения параметров вторичных цепей (мощности нагрузки) в системах учета электрической энергии. <<Энерготестер ПКЭ-А>>  проверяет работоспособность и правильность подключения однофазных и трехфазных счетчиков электрической энергии без разрыва токовых цепей.

\textbf{Научно-производственное предприятие <<Энерготехника>>}  (НПП <<Энерготехника>>, г.~Пенза) существует на рынке 26 лет. Основной изготавливаемой продукцией являются [73]: 
% НЕ РАБОТАЕТ САЙТ
%[73 - Научно-производственное предприятие «Энерготехника» [Электронный ресурс]. 2009-2019. Режим доступа: https://www.entp.ru/]

\begin{itemize}
	\item Измерители ПКЭ: <<Ресурс-UF2>>, <<Ресурс-UF2C>>, <<Ресурс-UF2М>>, <<Ресурс-ПКЭ>>.
	\item Счетчик многофункциональный <<Ресурс-Е4>>.
	\item Мультиметры: <<Ресурс-ПЭ>>, <<Ресурс-МТ>>.
	\item Калибраторы переменного тока <<Ресурс-К2>>,<<Ресурс-К2М>>.
\end{itemize}

Предприятие <<Энерготехника>> предоставляет измерение показателей качества электрической энергии, параметров напряжений, частоты, силы токов, активной и реактивной мощности, активной и реактивной энергии прямого и обратного направлений в трехфазных трехпроводных и четырехпроводных электрических сетях с помощью устройства Системы контроля качества электроэнергии <<Ресурс>> (СККЭ <<Ресурс>>). Устройство состоит из двух блоков: измерительного и  вычислительного.

<<Ресурс-UF2M-3Т52-5-100-1000>> измеритель показателей качества электрической энергии. Измерение ПКЭ производится согласно ГОСТ 30804.4.30–2013 \cite{ГОСТ30804.4.30-2013} (ГОСТ Р 51317.4.30–2008), класс А, ГОСТ 32144–2013 \cite{ГОСТ32144-2013} (ГОСТ Р 54149–2010). 

Прибор является мобильной версией прибора <<Ресурс‑UF2>>, обладает высокой точностью измерений силы тока без разрыва цепи с помощью токоизмерительных клещей (КТ52-5-100-1000 или КТ64-3000). Измеряет параметры напряжения, силы тока и угла фазового сдвига, мощности и энергии. <<Ресурс-UF2M-3Т52-5-100-1000>> записывает архивные данные на USB Flash-диск. Прибор регистрирует результаты измерений и аварийных событий, определяет выходную мощность измерительных трансформаторов напряжения, определяет погрешность счетчиков электрической энергии на месте эксплуатации. Прибор использует программное обеспечение: <<Ресурс-UF2 Opera>>, <<Ресурс-UF2 Plus>>, <<Ресурс-UF2 Plus>>, <<Монитор Ресурс-UF2>>, <<Ресурс-Бриз>>. 

ЗАО <<Научно-производственная фирма Энергосоюз>> (ЗАО <<НПФ Энергосоюз>>, г.~Санкт-Петербург)  c 1990 года специализируется на разработке, производстве и внедрении оборудования для автоматизации объектов электроэнергетики \cite{энергосоюз}.
% [69 - ЗАО «НПФ «ЭНЕРГОСОЮЗ» [Электронный ресурс]. 2009-2019. Режим доступа: http://www.energosoyuz.spb.ru/]

Один из первых разработок компании – цифровой регистратор аварийных событий Блок Регистрации, Контроля и Управления (БРКУ), зарекомендовавший как надежное, функциональное и простое в работе устройство. Продукция компании: Регистраторы аварийных событий, Телемеханика, Противоаварийная автоматика, Контроль и диагностика, Автоматика управления, Измерительные приборы, Серверное оборудование, ПО <<СКАДА-НЕВА>>.
Устройство противоаварийной автоматики <<НЕВА-ПА>> \cite{нева-па}. 
% Сайт прибора https://www.energosoyuz.spb.ru/ru/content/ustroystvo-protivoavariynoy-avtomatiki-neva-pa
Микропроцессорное устройство позволяет реализовать сложные алгоритмы противоаварийного управления: восстанавливать нормальное питание потребителей, выявлять и локализовать, развитие и аварийных режимов в энергосистемах, а также повышать пропускную способность электрических сетей. Основные функции устройства: предотвращать нарушение устойчивости, ликвидировать асинхронные режимы, ограничивать снижение или повышение частоты, ограничивать снижение или повышение напряжения, предотвращать перегрузку оборудования. Устройство <<НЕВА-ПА>> выполнено в блочном каркасе, имеет модульную структуру и снабжено для монтажа в несущую стоечную конструкцию типового шкафа управления.

Открытое акционерное общество (ОАО <<Электроприбор>>, г.~Чебоксары) – компания по производству щитовых аналоговых и цифровых электроизмерительных приборов, измерительныз преобразователей, приборов телемеханики, приборов для контроля ПКЭ. Основные напрвления компании: контрольно измерительные приборы и средства автоматизации.
<<ЩМК96>> прибор контроля КЭ – это современный многофункциональный анализатор КЭ, предназначенный для непрерывного измерения всех параметров трехфазных сетей переменного тока, а также ПКЭ и контроля их соответствия установленным нормам. Прибор спобен интегрироваться в различные системы телеизмерений, осуществляя одновременную передачу данных по нескольким направлениям. <<ЩМК96>> осуществляет мониторинг параметров электрической сети, непрерывный контроль и измерение КЭ, а также технический учет электроэнергии.

Предприятие НПФ <<Прорыв>> (г. Петрозаводск) является ведущим в России разработчиком и производителем испытательного оборудования и средств измерений в области электромагнитной совместимости \cite{прорыв}. 
% [208 - Научно-производственное предприятие «Прорыв» [Электронный ресурс]. 2000-2019. Режим доступа: https://proryvnpp.ru/]
Научно-производственное предприятие <<Прорыв>> создано 1991 году при Петрозаводском государственном университете. Уровень оборудования соответствует лучшим образцам зарубежных производителей при более низкой цене.

<<Прорыв-Т-А>> с токоизмерительными клещами <<Прорыв-КТ250>>.
% Сайт прибора https://proryvnpp.ru/product/proryv-t-a-s-tokoizmeritelnymi-kleshhami-proryv-kt250/
Предназназначение прибора для измерения и регистрации характеристик:
\begin{itemize}
	\item Напряжения.
	\item Силы тока.
	\item Реактивной мощности.
	\item Полной мощности.
	\item Временных характеристик.
	\item ПКЭ в соответствии с ГОСТ 32144-2013, ГОСТ 33073-2014, ГОСТ 30804.4.30-2013.
\end{itemize}

\begin{table} [p]%
	\caption{Технические характеристики <<Прорыв-Т-А>>.}%
	\label{tbl:test3}% label всегда желательно идти после caption
	\begin{SingleSpace}
		\setlength\extrarowheight{6pt} %вот этим управляем расстоянием между рядами, \arraystretch даёт неудачный результат
		\setlength{\tymin}{1.9cm}% минимальная ширина столбца
		\begin{tabulary}{\textwidth}{@{}>{\zz}L >{\zz}L @{}}% Вертикальные полосы не используются принципиально, как и лишние горизонтальные (допускается по ГОСТ 2.105 пункт 4.4.5) % @{} позволяет прижиматься к краям
			\toprule     %%% верхняя линейка
			Электропитание прибора осуществляется напряжением переменного тока в диапазоне & 
			от $85-265$В и частотов в диапазоне от $45-55$Гц\\
			
			Прибор обеспечивает непрерывное измерение и запоминание ПКЭ в течение &
			не менее $30$ суток \\
			
			Средний срок службы &
			не менее $10$ лет\\
			
			Прибор имеет наработку на отказ& не менее $70000$ часов \\
			
			%\midrule %%% тонкий разделитель. Отделяет названия столбцов. Обязателен по ГОСТ 2.105 пункт 4.4.5 
			
			\bottomrule %%% нижняя линейка
		\end{tabulary}%
	\end{SingleSpace}
\end{table}

%\chapter{Оформление различных элементов}\label{ch:ch1}
%\section{Форматирование текста}\label{sec:ch1/sec1}
%Мы можем сделать \textbf{жирный текст} и \textit{курсив}.
%\section{Ссылки}\label{sec:ch1/sec2}



Стандарты:
\cite{ГОСТ30804.4.7-2013}
\cite{ГОСТ30804.4.30-2013}
\cite{ГОСТ33073-2014}
\cite{ГОСТ32144-2013}
\cite{ГОСТР54149-2010}
\cite{ГОСТР51317.4.30-2008}
\cite{ГОСТР51317.4.7-2008}
\cite{ГОСТР8.655-2009}
\cite{ГОСТ13109-97}
\cite{ГОСТ13109-87}
\cite{ГОСТ8.216-88}
\cite{ГОСТ19431-84}
\cite{ГОСТ12.3.019-80}
\cite{ГОСТ21027-75}
\cite{ГОСТ16263-70}
\cite{ГОСТ13109-67}
\cite{ГОСТР51317.4.15-2012}
\cite{ГОСТ8.622-2013}



Ссылки на собственные работы:

\cite{альтман2019кэш}
\cite{васеева2019сравнительный}
\cite{васеева2019}
\cite{васеева2018информационно}
\cite{васееваисследование}
\cite{александров2018оценка}
\cite{васеева2018применение}
\cite{альтман2018расширение}
\cite{терентьева2017алгоритм}
\cite{альтман2017применение}
\cite{васеева2017распознавание}
\cite{Васеева2017}




\begin{figure}[ht]
	\centerfloat{
		\includegraphics[scale=0.7]{picture15}
	}
	\caption{Упрощенная классификация обычно используемых методов гармонических и межгармонических оценок.}\label{fig:picture2}
\end{figure}



%Сошлёмся на библиографию.
%Одна ссылка: \cite[с.~54]{Sokolov}\cite[с.~36]{Gaidaenko}.
%Две ссылки: \cite{Sokolov,Gaidaenko}.
%Ссылка на собственные работы: \cite{vakbib1, confbib2}.
%Много ссылок: %\cite[с.~54]{Lermontov,Management,Borozda} % такой «фокус»
%%вызывает biblatex warning относительно опции sortcites, потому что неясно, к
%%какому источнику относится уточнение о страницах, а bibtex об этой проблеме
%%даже не предупреждает
%\cite{Lermontov, Management, Borozda, Marketing, Constitution, FamilyCode,
%Gost.7.0.53, Razumovski, Lagkueva, Pokrovski, Methodology, Nasirova, Berestova,
%Kriger}%
%\ifnumequal{\value{bibliosel}}{0}{% Примеры для bibtex8
%    \cite{Sirotko, Lukina, Encyclopedia}%
%}{% Примеры для biblatex через движок biber
%    \cite{Sirotko2, Lukina2, Encyclopedia2}%
%}%
%.
%И~ещё немного ссылок:~\cite{Article,Book,Booklet,Conference,Inbook,Incollection,Manual,Mastersthesis,
%Misc,Phdthesis,Proceedings,Techreport,Unpublished}
%% Следует обратить внимание, что пробел после запятой внутри \cite{}
%% обрабатывается ожидаемо, а пробел перед запятой, может вызывать проблемы при
%% обработке ссылок.
%\cite{medvedev2006jelektronnye, CEAT:CEAT581, doi:10.1080/01932691.2010.513279,
%Gosele1999161,Li2007StressAnalysis, Shoji199895, test:eisner-sample,
%test:eisner-sample-shorted, AB_patent_Pomerantz_1968, iofis_patent1960}
%\ifnumequal{\value{bibliosel}}{0}{% Примеры для biblatex через движок biber
%    \cite{patent2h, patent3h, patent2}%
%}%
%.
%
%\ifnumequal{\value{bibliosel}}{0}{% Примеры для bibtex8
%Попытка реализовать несколько ссылок на конкретные страницы
%для \texttt{bibtex} реализации библиографии:
%[\citenum{Sokolov}, с.~54; \citenum{Gaidaenko}, с.~36].
%}{% Примеры для biblatex через движок biber
%Несколько источников (мультицитата):
%% Тут специально написано по-разному тире, для демонстрации, что
%% применение специальных тире в настоящий момент в biblatex приводит к непоказу
%% "с.".
%\cites[vii--x, 5, 7]{Sokolov}[v"--~x, 25, 526]{Gaidaenko}[vii--x, 5, 7]{Techreport},
%работает только в \texttt{biblatex} реализации библиографии.
%}%
%
%Ссылки на собственные работы:~\cite{vakbib1, confbib1}
%
%Сошлёмся на приложения: Приложение~\ref{app:A}, Приложение~\ref{app:B2}.
%
%Сошлёмся на формулу: формула~\eqref{eq:equation1}.
%
%Сошлёмся на изображение: рисунок~\ref{fig:knuth}.
%
%Стандартной практикой является добавление к ссылкам префикса, характеризующего тип элемента.
%Это не является строгим требованием, но~позволяет лучше ориентироваться в документах большого размера.
%Например, для ссылок на рисунки используется префикс \textit{fig},
%для ссылки на~таблицу "--- \textit{tab}.
%
%В таблице~\ref{tab:tab_pref} приложения~\ref{app:B4} приведён список рекомендуемых
%к использованию стандартных префиксов.
%
%%\section{Формулы}\label{sec:ch1/sec3}
%
%Благодаря пакету \textit{icomma}, \LaTeX~одинаково хорошо воспринимает
%в~качестве десятичного разделителя и запятую (\(3,1415\)), и точку (\(3.1415\)).
%
%\subsection{Ненумерованные одиночные формулы}\label{subsec:ch1/sec3/sub1}
%
%Вот так может выглядеть формула, которую необходимо вставить в~строку
%по~тексту: \(x \approx \sin x\) при \(x \to 0\).
%
%А вот так выглядит ненумерованная отдельностоящая формула c подстрочными
%и надстрочными индексами:
%\[
%(x_1+x_2)^2 = x_1^2 + 2 x_1 x_2 + x_2^2
%\]
%
%Формула с неопределенным интегралом:
%\[
%\int f(\alpha+x)=\sum\beta
%\]
%
%При использовании дробей формулы могут получаться очень высокие:
%\[
%  \frac{1}{\sqrt{2}+
%  \displaystyle\frac{1}{\sqrt{2}+
%  \displaystyle\frac{1}{\sqrt{2}+\cdots}}}
%\]
%
%В формулах можно использовать греческие буквы:
%%Все \original... команды заранее, ради этого примера, определены в Dissertation\userstyles.tex
%\[
%\alpha\beta\gamma\delta\originalepsilon\epsilon\zeta\eta\theta%
%\vartheta\iota\kappa\varkappa\lambda\mu\nu\xi\pi\varpi\rho\varrho%
%\sigma\varsigma\tau\upsilon\originalphi\phi\chi\psi\omega\Gamma\Delta%
%\Theta\Lambda\Xi\Pi\Sigma\Upsilon\Phi\Psi\Omega
%\]
%\[%https://texfaq.org/FAQ-boldgreek
%\boldsymbol{\alpha\beta\gamma\delta\originalepsilon\epsilon\zeta\eta%
%\theta\vartheta\iota\kappa\varkappa\lambda\mu\nu\xi\pi\varpi\rho%
%\varrho\sigma\varsigma\tau\upsilon\originalphi\phi\chi\psi\omega\Gamma%
%\Delta\Theta\Lambda\Xi\Pi\Sigma\Upsilon\Phi\Psi\Omega}
%\]
%
%Для добавления формул можно использовать пары \verb+$+\dots\verb+$+ и \verb+$$+\dots\verb+$$+,
%но~они считаются устаревшими.
%Лучше использовать их функциональные аналоги \verb+\(+\dots\verb+\)+ и \verb+\[+\dots\verb+\]+.
%
%\subsection{Ненумерованные многострочные формулы}\label{subsec:ch1/sec3/sub2}
%
%Вот так можно написать две формулы, не нумеруя их, чтобы знаки <<равно>> были
%строго друг под другом:
%\begin{align}
%  f_W & =  \min \left( 1, \max \left( 0, \frac{W_{soil} / W_{max}}{W_{crit}} \right)  \right), \nonumber \\
%  f_T & =  \min \left( 1, \max \left( 0, \frac{T_s / T_{melt}}{T_{crit}} \right)  \right), \nonumber
%\end{align}
%
%Выровнять систему ещё и по переменной \( x \) можно, используя окружение
%\verb|alignedat| из пакета \verb|amsmath|. Вот так:
%\[
%    |x| = \left\{
%    \begin{alignedat}{2}
%        &&x, \quad &\text{eсли } x\geqslant 0 \\
%        &-&x, \quad & \text{eсли } x<0
%    \end{alignedat}
%    \right.
%\]
%Здесь первый амперсанд (в исходном \LaTeX\ описании формулы) означает
%выравнивание по~левому краю, второй "--- по~\( x \), а~третий "--- по~слову
%<<если>>. Команда \verb|\quad| делает большой горизонтальный пробел.
%
%
%
%Ещё вариант:
%\[
%    |x|=
%    \begin{cases}
%    \phantom{-}x, \text{если } x \geqslant 0 \\
%    -x, \text{если } x<0
%    \end{cases}
%\]
%
%Кроме того, для  нумерованных формул \verb|alignedat| делает вертикальное
%выравнивание номера формулы по центру формулы. Например, выравнивание
%компонент вектора:
%\begin{equation}
%\label{eq:2p3}
%\begin{alignedat}{2}
%{\mathbf{N}}_{o1n}^{(j)} = \,{\sin} \phi\,n\!\left(n+1\right)
%         {\sin}\theta\,
%         \pi_n\!\left({\cos} \theta\right)
%         \frac{
%               z_n^{(j)}\!\left( \rho \right)
%              }{\rho}\,
%           &{\boldsymbol{\hat{\mathrm e}}}_{r}\,+   \\
%+\,
%{\sin} \phi\,
%         \tau_n\!\left({\cos} \theta\right)
%         \frac{
%            \left[\rho z_n^{(j)}\!\left( \rho \right)\right]^{\prime}
%              }{\rho}\,
%            &{\boldsymbol{\hat{\mathrm e}}}_{\theta}\,+   \\
%+\,
%{\cos} \phi\,
%         \pi_n\!\left({\cos} \theta\right)
%         \frac{
%            \left[\rho z_n^{(j)}\!\left( \rho \right)\right]^{\prime}
%              }{\rho}\,
%            &{\boldsymbol{\hat{\mathrm e}}}_{\phi}\:.
%\end{alignedat}
%\end{equation}
%
%Ещё об отступах. Иногда для лучшей <<читаемости>> формул полезно
%немного исправить стандартные интервалы \LaTeX\ с учётом логической
%структуры самой формулы. Например в формуле~\ref{eq:2p3} добавлен
%небольшой отступ \verb+\,+ между основными сомножителями, ниже
%результат применения всех вариантов отступа:
%\begin{align*}
%\backslash! &\quad f(x) = x^2\! +3x\! +2 \\
%  \mbox{по-умолчанию} &\quad f(x) = x^2+3x+2 \\
%\backslash, &\quad f(x) = x^2\, +3x\, +2 \\
%\backslash{:} &\quad f(x) = x^2\: +3x\: +2 \\
%\backslash; &\quad f(x) = x^2\; +3x\; +2 \\
%\backslash \mbox{space} &\quad f(x) = x^2\ +3x\ +2 \\
%\backslash \mbox{quad} &\quad f(x) = x^2\quad +3x\quad +2 \\
%\backslash \mbox{qquad} &\quad f(x) = x^2\qquad +3x\qquad +2
%\end{align*}
%
%Можно использовать разные математические алфавиты:
%\begin{align}
%\mathcal{ABCDEFGHIJKLMNOPQRSTUVWXYZ} \nonumber \\
%\mathfrak{ABCDEFGHIJKLMNOPQRSTUVWXYZ} \nonumber \\
%\mathbb{ABCDEFGHIJKLMNOPQRSTUVWXYZ} \nonumber
%\end{align}
%
%Посмотрим на систему уравнений на примере аттрактора Лоренца:
%
%\[
%\left\{
%  \begin{array}{rl}
%    \dot x = & \sigma (y-x) \\
%    \dot y = & x (r - z) - y \\
%    \dot z = & xy - bz
%  \end{array}
%\right.
%\]
%
%А для вёрстки матриц удобно использовать многоточия:
%\[
%\left(
%  \begin{array}{ccc}
%    a_{11} & \ldots & a_{1n} \\
%    \vdots & \ddots & \vdots \\
%    a_{n1} & \ldots & a_{nn} \\
%  \end{array}
%\right)
%\]
%
%\subsection{Нумерованные формулы}\label{subsec:ch1/sec3/sub3}
%
%А вот так пишется нумерованная формула:
%\begin{equation}
%  \label{eq:equation1}
%  e = \lim_{n \to \infty} \left( 1+\frac{1}{n} \right) ^n
%\end{equation}
%
%Нумерованных формул может быть несколько:
%\begin{equation}
%  \label{eq:equation2}
%  \lim_{n \to \infty} \sum_{k=1}^n \frac{1}{k^2} = \frac{\pi^2}{6}
%\end{equation}
%
%Впоследствии на формулы~\eqref{eq:equation1} и~\eqref{eq:equation2} можно ссылаться.
%
%Сделать так, чтобы номер формулы стоял напротив средней строки, можно,
%используя окружение \verb|multlined| (пакет \verb|mathtools|) вместо
%\verb|multline| внутри окружения \verb|equation|. Вот так:
%\begin{equation} % \tag{S} % tag - вписывает свой текст
%  \label{eq:equation3}
%    \begin{multlined}
%        1+ 2+3+4+5+6+7+\dots + \\
%        + 50+51+52+53+54+55+56+57 + \dots + \\
%        + 96+97+98+99+100=5050
%    \end{multlined}
%\end{equation}
%
%Используя команду \verb|\labelcref| из пакета \verb|cleveref|, можно
%красиво ссылаться сразу на несколько формул
%(\labelcref{eq:equation1, eq:equation3, eq:equation2}), даже перепутав
%порядок ссылок \verb|(\labelcref{eq:equation1, eq:equation3, eq:equation2})|.
%
%Уравнения~(\labelcref{eq:subeq_1,eq:subeq_2}) демонстрируют возможности
%окружения \verb|\subequations|.
%\begin{subequations}
%	\label{eq:subeq_1}
%	\begin{gather}
%		y = x^2 + 1 \label{eq:subeq_1-1} \\
%		y = 2 x^2 - x + 1 \label{eq:subeq_1-2}
%	\end{gather}
%\end{subequations}
%Ссылки на отдельные уравнения~(\labelcref{eq:subeq_1-1,eq:subeq_1-2,eq:subeq_2-1}).
%\begin{subequations}
%	\label{eq:subeq_2}
%	\begin{align}
%		y &= x^3 + x^2 + x + 1 \label{eq:subeq_2-1} \\
%		y &= x^2
%	\end{align}
%\end{subequations}
%
%\subsection{Форматирование чисел и размерностей величин}\label{sec:units}
%
%Числа форматируются при помощи команды \verb|\num|:
%\num{5,3};
%\num{2,3e8};
%\num{12345,67890};
%\num{2,6 d4};
%\num{1+-2i};
%\num{.3e45};
%\num[exponent-base=2]{5 e64};
%\num[exponent-base=2,exponent-to-prefix]{5 e64};
%\num{1.654 x 2.34 x 3.430}
%\num{1 2 x 3 / 4}.
%Для написания последовательности чисел можно использовать команды \verb|\numlist| и \verb|\numrange|:
%\numlist{10;30;50;70}; \numrange{10}{30}.
%Значения углов можно форматировать при помощи команды \verb|\ang|:
%\ang{2.67};
%\ang{30,3};
%\ang{-1;;};
%\ang{;-2;};
%\ang{;;-3};
%\ang{300;10;1}.
%
%Обратите внимание, что ГОСТ запрещает использование знака <<->> для обозначения отрицательных чисел
%за исключением формул, таблиц и рисунков.
%Вместо него следует использовать слово <<минус>>.
%
%Размерности можно записывать при помощи команд \verb|\si| и \verb|\SI|:
%\si{\farad\squared\lumen\candela};
%\si{\joule\per\mole\per\kelvin};
%\si[per-mode = symbol-or-fraction]{\joule\per\mole\per\kelvin};
%\si{\metre\per\second\squared};
%\SI{1.2-3i e5}{\joule\per\mole\per\kelvin};
%\SI{0.10(5)}{\neper};
%\SIlist{1;2;3;4}{\tesla};
%\SIrange{50}{100}{\volt}.
%Список единиц измерений приведён в таблицах~(\labelcref{tab:unit:base,
%tab:unit:derived,tab:unit:accepted,tab:unit:physical,tab:unit:other}).
%Приставки единиц приведены в таблице~\ref{tab:unit:prefix}.
%
%С дополнительными опциями форматирования можно ознакомиться в описании пакета \texttt{siunitx};
%изменить или добавить единицы измерений можно в~файле \texttt{siunitx.cfg}.
%
%\begin{table}
%    \caption{Основные величины СИ.}\label{tab:unit:base}
%    \centering
%    \begin{tabular}{llc}
%        \toprule
%        Название  & Команда                & Символ         \\
%        \midrule
%        Ампер     & \verb|\ampere| & \si{\ampere}   \\
%        Кандела   & \verb|\candela| & \si{\candela}  \\
%        Кельвин   & \verb|\kelvin| & \si{\kelvin}   \\
%        Килограмм & \verb|\kilogram| & \si{\kilogram} \\
%        Метр      & \verb|\metre| & \si{\metre}    \\
%        Моль      & \verb|\mole| & \si{\mole}     \\
%        Секунда   & \verb|\second| & \si{\second}   \\
%        \bottomrule
%    \end{tabular}
%\end{table}
%
%\begin{table}
%    \caption{Производные единицы СИ.}\label{tab:unit:derived}
%    \small
%    \centering
%    \begin{tabular}{llc|llc}
%        \toprule
%        Название       & Команда                 & Символ              & Название & Команда & Символ \\
%        \midrule
%        Беккерель      & \verb|\becquerel|  & \si{\becquerel}     &
%        Ньютон         & \verb|\newton|  & \si{\newton}                                      \\
%        Градус Цельсия & \verb|\degreeCelsius| & \si{\degreeCelsius} &
%        Ом             & \verb|\ohm| & \si{\ohm}                                         \\
%        Кулон          & \verb|\coulomb| & \si{\coulomb}       &
%        Паскаль        & \verb|\pascal| & \si{\pascal}                                      \\
%        Фарад          & \verb|\farad| & \si{\farad}         &
%        Радиан         & \verb|\radian| & \si{\radian}                                      \\
%        Грей           & \verb|\gray| & \si{\gray}          &
%        Сименс         & \verb|\siemens| & \si{\siemens}                                     \\
%        Герц           & \verb|\hertz| & \si{\hertz}         &
%        Зиверт         & \verb|\sievert| & \si{\sievert}                                     \\
%        Генри          & \verb|\henry| & \si{\henry}         &
%        Стерадиан      & \verb|\steradian| & \si{\steradian}                                   \\
%        Джоуль         & \verb|\joule| & \si{\joule}         &
%        Тесла          & \verb|\tesla| & \si{\tesla}                                       \\
%        Катал          & \verb|\katal| & \si{\katal}         &
%        Вольт          & \verb|\volt| & \si{\volt}                                        \\
%        Люмен          & \verb|\lumen| & \si{\lumen}         &
%        Ватт           & \verb|\watt| & \si{\watt}                                        \\
%        Люкс           & \verb|\lux| & \si{\lux}           &
%        Вебер          & \verb|\weber| & \si{\weber}                                       \\
%        \bottomrule
%    \end{tabular}
%\end{table}
%
%\begin{table}
%    \caption{Внесистемные единицы.}\label{tab:unit:accepted}
%    \centering
%    \begin{tabular}{llc}
%        \toprule
%        Название        & Команда                 & Символ          \\
%        \midrule
%        День            & \verb|\day| & \si{\day}       \\
%        Градус          & \verb|\degree| & \si{\degree}    \\
%        Гектар          & \verb|\hectare| & \si{\hectare}   \\
%        Час             & \verb|\hour| & \si{\hour}      \\
%        Литр            & \verb|\litre| & \si{\litre}     \\
%        Угловая минута  & \verb|\arcminute| & \si{\arcminute} \\
%        Угловая секунда & \verb|\arcsecond| & \si{\arcsecond} \\ %
%        Минута          & \verb|\minute| & \si{\minute}    \\
%        Тонна           & \verb|\tonne| & \si{\tonne}     \\
%        \bottomrule
%    \end{tabular}
%\end{table}
%
%\begin{table}
%    \caption{Внесистемные единицы, получаемые из эксперимента.}\label{tab:unit:physical}
%    \centering
%    \begin{tabular}{llc}
%        \toprule
%        Название                & Команда                 & Символ                 \\
%        \midrule
%        Астрономическая единица & \verb|\astronomicalunit| & \si{\astronomicalunit} \\
%        Атомная единица массы   & \verb|\atomicmassunit| & \si{\atomicmassunit}   \\
%        Боровский радиус        & \verb|\bohr| & \si{\bohr}             \\
%        Скорость света          & \verb|\clight| & \si{\clight}           \\
%        Дальтон                 & \verb|\dalton| & \si{\dalton}           \\
%        Масса электрона         & \verb|\electronmass| & \si{\electronmass}     \\
%        Электрон Вольт          & \verb|\electronvolt| & \si{\electronvolt}     \\
%        Элементарный заряд      & \verb|\elementarycharge| & \si{\elementarycharge} \\
%        Энергия Хартри          & \verb|\hartree| & \si{\hartree}          \\
%        Постоянная Планка       & \verb|\planckbar| & \si{\planckbar}        \\
%        \bottomrule
%    \end{tabular}
%\end{table}
%
%\begin{table}
%    \caption{Другие внесистемные единицы.}\label{tab:unit:other}
%    \centering
%    \begin{tabular}{llc}
%        \toprule
%        Название                  & Команда                 & Символ             \\
%        \midrule
%        Ангстрем                  & \verb|\angstrom| & \si{\angstrom}     \\
%        Бар                       & \verb|\bar| & \si{\bar}          \\
%        Барн                      & \verb|\barn| & \si{\barn}         \\
%        Бел                       & \verb|\bel| & \si{\bel}          \\
%        Децибел                   & \verb|\decibel| & \si{\decibel}      \\
%        Узел                      & \verb|\knot| & \si{\knot}         \\
%        Миллиметр ртутного столба & \verb|\mmHg| & \si{\mmHg}         \\
%        Морская миля              & \verb|\nauticalmile| & \si{\nauticalmile} \\
%        Непер                     & \verb|\neper| & \si{\neper}        \\
%        \bottomrule
%    \end{tabular}
%\end{table}
%
%\begin{table}
%    \caption{Приставки СИ.}\label{tab:unit:prefix}
%    \centering
%    \small
%    \begin{tabular}{llcc|llcc}
%        \toprule
%        Приставка & Команда                 & Символ      & Степень &
%        Приставка & Команда                 & Символ      & Степень   \\
%        \midrule
%        Иокто     & \verb|\yocto| & \si{\yocto} & -24     &
%        Дека      & \verb|\deca| & \si{\deca}  & 1         \\
%        Зепто     & \verb|\zepto| & \si{\zepto} & -21     &
%        Гекто     & \verb|\hecto| & \si{\hecto} & 2         \\
%        Атто      & \verb|\atto| & \si{\atto}  & -18     &
%        Кило      & \verb|\kilo| & \si{\kilo}  & 3         \\
%        Фемто     & \verb|\femto| & \si{\femto} & -15     &
%        Мега      & \verb|\mega| & \si{\mega}  & 6         \\
%        Пико      & \verb|\pico| & \si{\pico}  & -12     &
%        Гига      & \verb|\giga| & \si{\giga}  & 9         \\
%        Нано      & \verb|\nano| & \si{\nano}  & -9      &
%        Терра     & \verb|\tera| & \si{\tera}  & 12        \\
%        Микро     & \verb|\micro| & \si{\micro} & -6      &
%        Пета      & \verb|\peta| & \si{\peta}  & 15        \\
%        Милли     & \verb|\milli| & \si{\milli} & -3      &
%        Екса      & \verb|\exa| & \si{\exa}   & 18        \\
%        Санти     & \verb|\centi| & \si{\centi} & -2      &
%        Зетта     & \verb|\zetta| & \si{\zetta} & 21        \\
%        Деци      & \verb|\deci| & \si{\deci}  & -1      &
%        Иотта     & \verb|\yotta| & \si{\yotta} & 24        \\
%        \bottomrule
%    \end{tabular}
%\end{table}

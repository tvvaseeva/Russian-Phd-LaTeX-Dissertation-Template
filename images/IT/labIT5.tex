\documentclass[11pt]{article}
\usepackage{lipsum}
\usepackage[utf8x]{inputenc}
\usepackage[russian]{babel}
\setlength{\parindent}{0pt}
\usepackage{tikz}
\usepackage{pgfplots}
\pgfplotsset{width=10cm,compat=1.12}
\definecolor{myblue}{HTML}{08088A} 
\begin{document}
	\begin{tikzpicture}
	\begin{axis}[
	title=Двухтактовое окно для интергармоники 90 Гц,
	width=13cm,
	height=5cm,
	ybar,
	ytick distance=0.2,
	xtick distance=30,
	xmin=0,
	xmax=150,
	ymin=0,
	ymax=1,
	enlarge x limits=0.125,
	bar width=35pt,
	y grid style={densely dotted, line cap=round},
	ylabel={Амплитуда},
	]
	\addplot [fill=myblue] coordinates { 
		(60,1)
		(90,0.5)
	};
	\end{axis}
	\node at (12,2.8) {а)};
	\end{tikzpicture} 
	\begin{tikzpicture}
	\begin{axis}[
	title=Двухтактовое окно для интергармоники 100 Гц,
	width=13cm,
	height=5cm,
	ybar,
	ytick distance=0.2,
	xtick distance=30,
	xmin=0,
	xmax=180,
	ymin=0,
	ymax=1,
	enlarge x limits=0.125,
	bar width=35pt,
	ylabel={Амплитуда},
	xlabel={Частота (Гц)}
	]
	\addplot [fill=myblue] coordinates { 
		(0,0.08)
		(30,0.1)
		(60,1)
		(90,0.5)
		(120,0.2)
		(150,0.1)
		(180,0.06)
	};
	\end{axis}
	\node at (12,2.8) {б)};
	\end{tikzpicture}
	\begin{tikzpicture}
	\node at (-10,-6) {Рис. 1 спектр промежуточных гармоник. а) синхронный анализ.};
	\node at (-12,-6.5) {б) конструкция-хронологический анализ.};
	\end{tikzpicture}
\end{document}
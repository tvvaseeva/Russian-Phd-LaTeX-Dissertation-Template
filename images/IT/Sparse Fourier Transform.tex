\documentclass{proc}
\usepackage[utf8]{inputenc}
\usepackage[english,russian]{babel}
\usepackage{tikz}
\usepackage{float}
%\documentclass{article}
\usepackage{pgfplots}
\usetikzlibrary{snakes}
\usetikzlibrary{plotmarks}
\usepackage[russian,english]{babel}
\usepackage{mathtext}
\usepackage{amsmath,amssymb,amsfonts,textcomp,latexsym,pb-diagram,amsopn}\usepackage[utf8]{inputenc}
\pgfplotsset{compat=newest}
\begin{document}
\begin{center}
\begin{tikzpicture}
\draw (0,0) -- (7,0);

\node[left,scale=1,font=\small] at (0.6,7){\large{ВРЕМЯ}};
\node[left,scale=1,font=\small] at (0.8,6.6){\large{ВЫПОЛ-}};
\node[left,scale=1,font=\small] at (0.9,6.2){\large{НЕНИЯ [s]}};	


% Значения осей
\draw[dashed] (0,0)node[left,font=\tiny] {\large{$10^-4$}};
\draw[dashed] (0,1)node[left,font=\tiny] {\large{$10^-3$}};
\draw[dashed] (0,2)node[left,font=\tiny] {\large{$10^-2$}};
\draw[dashed] (0,3)node[left,font=\tiny] {\large{$10^-1$}};
\draw[dashed] (0,4)node[left,font=\tiny] {\large{$10^0$}};
\draw[dashed] (0,5)node[left,font=\tiny] {\large{$10^1$}};


\draw[dashed] (1,0)node[below,font=\tiny] {\large{$2^6$}};
\draw[dashed] (2,0)node[below,font=\tiny] {\large{$2^7$}};
\draw[dashed] (3,0)node[below,font=\tiny] {\large{$2^8$}};
\draw[dashed] (4,0)node[below,font=\tiny] {\large{$2^9$}};
\draw[dashed] (5,0)node[below,font=\tiny] {\large{$2^{10}$}};
\draw[dashed] (6,0)node[below,font=\tiny] {\large{$2^{11}$}};
\draw[dashed] (7,0)node[below,font=\tiny] {\large{$2^{12}$}};




\node[below,font=\tiny] at (0,0){\large{$2^5$}};
\node[below,scale=1, font=\small] at (3.5,-0.5){\large{КОЛИЧЕСТВО НЕНУЛЕВЫХ }};
\node[below,scale=1, font=\small] at (3.5,-1){\large{ЧАСТОТ k}};


% Оси
\draw[->] (0,0) -- (7,0) coordinate (x axis);
\draw[->] (0,0) -- (0,6) coordinate (y axis);
%График

\draw[black] (1,0) -- (1,0.08);
\draw[black] (2,0) -- (2,0.08);
\draw[black] (3,0) -- (3,0.08);
\draw[black] (4,0) -- (4,0.08);
\draw[black] (5,0) -- (5,0.08);
\draw[black] (6,0) -- (6,0.08);

\draw[black] (0,1) -- (0.08,1);
\draw[black] (0,2) -- (0.08,2);
\draw[black] (0,3) -- (0.08,3);
\draw[black] (0,4) -- (0.08,4);
\draw[black] (0,5) -- (0.08,5);

% FFTW
\draw[ultra thick, red] (0.6,3.3) -- (7,3.3);
\draw[ultra thick, red] (0.6,3.6) -- (7,3.6);

%SFFT v1
\draw[ultra thick, blue] (0.6,2.2) -- (1.7,2.7);
\draw[ultra thick, blue] (1.7,2.7) -- (2.7,3);
\draw[ultra thick, blue] (2.7,3) -- (4,3.2);
\draw[ultra thick, blue] (4,3.2) -- (5,3.8);
\draw[ultra thick, blue] (5,3.8) -- (6,3.8);
\draw[ultra thick, blue] (6,3.8) -- (6.4,4.1);
\draw[ultra thick, blue] (6.4,4.1) -- (7,4.8);

%SFFT v3
\draw[ultra thick, magenta] (0.6,0.3) -- (1.8,0.5);
\draw[ultra thick, magenta] (1.8,0.5) -- (2.8,0.8);
\draw[ultra thick, magenta] (2.8,0.8) -- (4,1);
\draw[ultra thick, magenta] (4,1) -- (5,1.3);
\draw[ultra thick, magenta] (5,1.3) -- (6,1.7);
\draw[ultra thick, magenta] (6,1.7) -- (6.4,1.8);
\draw[ultra thick, magenta] (6.4,1.8) -- (7,2);

%легенда
\draw[ultra thick, red] (1, 6.3) -- (1.5, 6.3) node[right, black, font=\small] {-- \large{FFTW (МЕРА FFTW}};
\node[right, black, font=\small] at (1.8, 5.8){\large{И ОЦЕНКА FFTW)}};
\draw[ultra thick, blue] (1, 5.3) -- (1.5, 5.3) node[right, black, font=\small] {-- \large{SFFT v1}};
\draw[ultra thick, magenta] (1, 4.8) -- (1.5, 4.8) node[right, black, font=\small] {-- \large{SFFT v3}};

\end{tikzpicture}
\end{center}
\end{document}

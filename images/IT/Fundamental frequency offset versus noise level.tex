\documentclass{proc}
\usepackage[utf8]{inputenc}
\usepackage[english,russian]{babel}
\usepackage{tikz}
\usepackage{float}
%\documentclass{article}
\usepackage{pgfplots}
\usetikzlibrary{snakes}
\usetikzlibrary{plotmarks}
\usepackage[russian,english]{babel}
\usepackage{mathtext}
\usepackage{amsmath,amssymb,amsfonts,textcomp,latexsym,pb-diagram,amsopn}\usepackage[utf8]{inputenc}
\pgfplotsset{compat=newest}
\begin{document}
\begin{center}
\begin{tikzpicture}
\draw (0,0) -- (8,0);
\node[right,scale=1,font=\small] at (0,6){\large{Смещение,Гц}};
% Значения осей
\draw[dashed] (0,1)node[left,font=\tiny] {\large{$0.02$}};
\draw[dashed] (0,2)node[left,font=\tiny] {\large{$0.04$}};
\draw[dashed] (0,3)node[left,font=\tiny] {\large{$0.06$}};
\draw[dashed] (0,4)node[left,font=\tiny] {\large{$0.08$}};
\draw[dashed] (0,5)node[left,font=\tiny] {\large{$0.1$}};
\draw[dashed] (0,0)node[below,font=\tiny] {\large{$-10$}};
\draw[dashed] (1,0)node[below,font=\tiny] {\large{$0$}};
\draw[dashed] (2,0)node[below,font=\tiny] {\large{$10$}};
\draw[dashed] (3,0)node[below,font=\tiny] {\large{$20$}};
\draw[dashed] (4,0)node[below,font=\tiny] {\large{$30$}};
\draw[dashed] (5,0)node[below,font=\tiny] {\large{$40$}};
\draw[dashed] (6,0)node[below,font=\tiny] {\large{$50$}};
\draw[dashed] (7,0)node[below,font=\tiny] {\large{$60$}};

\node[left,font=\tiny] at (0,0){\large{$0$}};
\node[above,scale=1, font=\small] at (7.85,0.1){\large{SNR,дБ}};
% Оси
\draw[->] (0,0) -- (8,0) coordinate (x axis);
\draw[->] (0,0) -- (0,6) coordinate (y axis);
%График

% алгоритм параболическоц интерполяции
\draw[ultra thick, magenta] (0,5) -- (0.5,2.8);
\draw[ultra thick, magenta] (0.5,2.8) -- (1,1.7);
\draw[ultra thick, magenta] (1,1.7) -- (1.5,1.1);
\draw[ultra thick, magenta] (1.5,1.1) -- (2,0.7);
\draw[ultra thick, magenta] (2,0.7) -- (2.5,0.6);
\draw[ultra thick, magenta] (2.5,0.6) -- (3,0.6);
\draw[ultra thick, magenta] (3,0.6) -- (3.5,0.6);
\draw[ultra thick, magenta] (3.5,0.6) -- (4,0.6);
\draw[ultra thick, magenta] (4,0.6) -- (4.5,0.6);
\draw[ultra thick, magenta] (4.5,0.6) -- (5,0.6);
\draw[ultra thick, magenta] (5,0.6) -- (5.5,0.6);
\draw[ultra thick, magenta] (5.5,0.6) -- (6,0.6);
\draw[ultra thick, magenta] (6,0.6) -- (6.5,0.6);
\draw[ultra thick, magenta] (6.5,0.6) -- (7,0.6);

% алгоритм интерполяции Гаусса
\draw[ultra thick,cyan] (0,5.1) -- (0.5,2.7);;
\draw[ultra thick,cyan] (0.5,2.7) -- (1,1.8);
\draw[ultra thick,cyan] (1,1.8) -- (1.5,0.9);
\draw[ultra thick,cyan] (1.5,0.9) -- (2,0.5);
\draw[ultra thick,cyan] (2,0.5) -- (2.5,0.4);
\draw[ultra thick,cyan] (2.5,0.4) -- (3,0.3);
\draw[ultra thick,cyan] (3,0.3) -- (3.5,0.2);
\draw[ultra thick,cyan] (3.5,0.2) -- (4,0.1);
\draw[ultra thick,cyan] (4,0.1) -- (4.5,0);
\draw[ultra thick,cyan] (4.5,0) -- (5,0);
\draw[ultra thick,cyan] (5,0) -- (5.5,0);
\draw[ultra thick,cyan] (5.5,0) -- (6,0);
\draw[ultra thick,cyan] (6,0) -- (6.5,0);
\draw [ultra thick,cyan] (6.5,0) -- (7,0);

% метод Грэндка
\draw[ultra thick, red] (0,2.7) -- (0.5,1.4);
\draw[ultra thick, red] (0.5,1.4) -- (1,0.8);
\draw[ultra thick, red] (1,0.8) -- (1.5,0.4);
\draw[ultra thick, red] (1.5,0.4) -- (2,0.3);
\draw[ultra thick, red] (2,0.3) -- (2.5,0.25);
\draw[ultra thick, red] (2.5,0.25) -- (3,0.15);
\draw[ultra thick, red] (3,0.15) -- (3.5,0.1);
\draw[ultra thick, red] (3.5,0.1) -- (4,0.05);
\draw[ultra thick, red] (4,0.05) -- (4.5,0);
\draw[ultra thick, red] (4.5,0) -- (5,0);
\draw[ultra thick, red] (5,0) -- (5.5,0);
\draw[ultra thick, red] (5.5,0) -- (6,0);
\draw[ultra thick, red] (6,0) -- (6.5,0);
\draw[ultra thick, red] (6.5,0) -- (7,0);

% метод Якобсена
\draw[ultra thick, blue] (0,1.2) -- (0.5,0.8);
\draw[ultra thick, blue] (0.5,0.8) -- (1,0.5);
\draw[ultra thick, blue] (1,0.5) -- (1.5,0.3);
\draw[ultra thick, blue] (1.5,0.3) -- (2,0.25);
\draw[ultra thick, blue] (2,0.25) -- (2.5,0.2);
\draw[ultra thick, blue] (2.5,0.2) -- (3,0.15);
\draw[ultra thick, blue] (3,0.15) -- (3.5,0.1);
\draw[ultra thick, blue] (3.5,0.1) -- (4,0.05);
\draw[ultra thick, blue] (4,0.05) -- (4.5,0.02);
\draw[ultra thick, blue] (4.5,0) -- (5,0);
\draw[ultra thick, blue] (5,0) -- (5.5,0);
\draw[ultra thick, blue] (5.5,0) -- (6,0);
\draw[ultra thick, blue] (6,0) -- (6.5,0);
\draw[ultra thick, blue] (6.5,0) -- (7,0);

% метод Макледона
\draw[ultra thick, green] (0,1.1) -- (0.5,0.7);
\draw[ultra thick, green] (0.5,0.7) -- (1,0.4);
\draw[ultra thick, green] (1,0.4) -- (1.5,0.25);
\draw[ultra thick, green] (1.5,0.25) -- (2,0.2);
\draw[ultra thick, green] (2,0.2) -- (2.5,0.2);
\draw[ultra thick, green] (2.5,0.2) -- (3,0.15);
\draw[ultra thick, green] (3,0.15) -- (3.5,0.1);
\draw[ultra thick, green] (3.5,0.1) -- (4,0.05);
\draw[ultra thick, green] (4,0.05) -- (4.5,0.02);
%\draw[ultra thick, green] (4.5,0) -- (5,0);
%\draw[ultra thick, green] (5,0) -- (5.5,0);
%\draw[ultra thick, green] (5.5,0) -- (6,0);
%\draw[ultra thick, green] (6,0) -- (6.5,0);
%\draw[ultra thick, green] (6.5,0) -- (7,0);

% второй метод Квина
\draw[ultra thick, black!50] (0,1) -- (0.5,0.6);
\draw[ultra thick, black!50] (0.5,0.6) -- (1,0.3);
\draw[ultra thick, black!50] (1,0.3) -- (1.5,0.15);
\draw[ultra thick, black!50] (1.5,0.15) -- (2,0.13);
\draw[ultra thick, black!50] (2,0.13) -- (2.5,0.1);
\draw[ultra thick, black!50] (2.5,0.1) -- (3,0.05);
\draw[ultra thick, black!50] (3,0.05) -- (3.5,0);

% метод корреляционных функций
\draw[ultra thick, black] (0,0.2) -- (0.5,0.15);
\draw[ultra thick, black] (0.5,0.15) -- (1,0.1);
\draw[ultra thick, black] (1,0.1) -- (1.5,0.03);
\draw[ultra thick, black] (1.5,0.03) -- (2,0.02);
\draw[ultra thick, black] (2,0.02) -- (2.5,0.01);
\draw[ultra thick, black] (2.5,0) -- (3,0);
\draw[ultra thick, black] (3,0) -- (3.5,0);
\draw[ultra thick, black] (3.5,0) -- (4,0);
\draw[ultra thick, black] (4,0) -- (4.5,0);
\draw[ultra thick, black] (4.5,0) -- (5,0);
\draw[ultra thick, black] (5,0) -- (5.5,0);
\draw[ultra thick, black] (5.5,0) -- (6,0);
\draw[ultra thick, black] (6,0) -- (6.5,0);
\draw[ultra thick, black] (6.5,0) -- (7,0);

% Легенда
\draw[ultra thick, black] (3, 5.7) -- (3.5, 5.7) node[right, black, font=\small] {-- \Large{метод корреляционных функций}};
\draw[ultra thick, blue]  (3, 5.3) -- (3.5, 5.3) node[right, black, font=\small] {-- \Large{метод Якобсена}};
\draw[ultra thick, green] (3, 4.9) -- (3.5, 4.9) node[right, black, font=\small] {-- \Large{метод Макледона}};
\draw[ultra thick, black!50] (3, 4.5) -- (3.5, 4.5) node[right, black, font=\small] {-- \Large{второй метод Квина}};
\draw[ultra thick, red] (3, 4.1) -- (3.5, 4.1) node[right, black, font=\small] {-- \Large{метод Грэндка}};
\draw[ultra thick, magenta] (3, 3.7) -- (3.5, 3.7) node[right, black, font=\small] {-- \Large{алгоритм параболической}};
\node[right, black, font=\small] at (3.75, 3.3){\Large{интерполяции}};
\draw[ultra thick,cyan] (3, 2.9) -- (3.5, 2.9) node[right, black, font=\small] {-- \Large{алгоритм интерполяции Гаусса}};
\end{tikzpicture}
\end{center}
\end{document}

% Преамбула
\documentclass{proc}
\usepackage{mathtext}
\usepackage[T1,T2A]{fontenc}
\usepackage[utf8]{inputenc}
\usepackage[english,russian]{babel}
\usepackage{tikz}
\usepackage{float}
%\documentclass{article}
\usepackage{pgfplots}
\usepackage[papersize={30cm,29.7cm}]{geometry}
\usetikzlibrary{snakes}
\usepackage{amsmath,amssymb,amsfonts,textcomp,latexsym,pb-diagram,amsopn}\usepackage[utf8]{inputenc}
\pgfplotsset{compat=newest}
\pagestyle{empty} % Отключить нумерацию странци

\begin{document}
\begin{center}
\begin{tikzpicture}[scale=3,decoration={brace,raise=5pt}]
\begin{Large}
% Нумерация по x
\node[left] at (0,0){$0$};
\node[left] at (0,0.5){$0.1$};
\node[left] at (0,1){$0.2$};
\node[left] at (0,1.5){$0.3$};
\node[left] at (0,2){$0.4$};
\node[left] at (0,2.5){$0.5$};
\node[left] at (0,3){$0.6$};
\node[left] at (0,3.5){$0.7$};
\node[left] at (0,4){$0.8$};
\node[left] at (0,4.5){$0.9$};
\node[left] at (0,5){$1$};

% Нумерация по y
\node[below] at (0.5,0){$200$};
\node[below] at (1,0){$400$};
\node[below] at (1.5,0){$600$};
\node[below] at (2,0){$800$};
\node[below] at (2.5,0){$1000$};
\node[below] at (3,0){$1200$};
\node[below] at (3.5,0){$1400$};
\node[below] at (4,0){$1600$};
\node[below] at (4.5,0){$1800$};
\node[below] at (5,0){$2000$};
\end{Large}
% Оси и сетка
\draw[->] (0,0) -- (5,0) coordinate (x axis) node[above] {\LARGE{$t$}};
\draw[->] (0,0) -- (0,5) coordinate (y axis) node[right] {\LARGE{$w(t)$}};

% Наименования графиков
\draw[ultra thick, red] (4.1, 4.9) sin (4.3, 4.9) node[right, black,] {\LARGE{$\beta = \ 4$}};
\draw[ultra thick, blue] (4.1, 4.6) sin (4.3, 4.6) node[right, black] {\LARGE{$\beta = \ 9$}};
\draw[ultra thick, violet] (4.1, 4.3) sin (4.3, 4.3) node[right, black] {\LARGE{$\beta = \ 10$}};
\draw[ultra thick, green] (4.1, 4) sin (4.3, 4) node[right, black] {\LARGE{$\beta  = \ 15$}};


% Графики
\draw[ultra thick, red] (0,0.5) sin (2.5,5) cos (5,0.5); % Кайзера 4
\draw[ultra thick, blue] (0,0) cos (1,1) sin (2.5,5) cos (4,1) sin (5,0); % Кайзера 9
\draw[ultra thick, violet] (0,0) cos (1,0.8) sin (2.5,5) cos (4,0.8) sin (5,0); % Кайзера 10
\draw[ultra thick, green] (0,0) cos (1.3,0.8) sin (2.5,5) cos (3.7,0.8) sin (5,0); % Кайзера 15
\end{tikzpicture}
\end{center}
\end{document}
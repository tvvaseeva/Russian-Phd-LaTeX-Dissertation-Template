\documentclass{proc}
\usepackage[utf8]{inputenc}
\usepackage[english,russian]{babel}
\usepackage{tikz}
\usepackage{float}
\usepackage{pgfplots}
\usepackage[papersize={20cm,20cm}]{geometry}
\usetikzlibrary{snakes}
\usepackage[russian,english]{babel}
\usepackage{mathtext}  
\usepackage{amsmath,amssymb,amsfonts,textcomp,latexsym,pb-diagram,amsopn}\usepackage[utf8]{inputenc}
\pgfplotsset{compat=newest}
\pagestyle{empty} % Отключает нумерацию страницы
\begin{document}
\begin{tikzpicture}[sibling distance=23em,
		every node/.style = {shape=rectangle, rounded corners,
			draw, align=center,
			top color=white, bottom color=blue!30}]]
		\node {\textbf{Быстрое}\\ \textbf{вычисление} \\
				\textbf{2D корреляций} \\ \textbf{и сверток}}
		child { node {\textbf{На основе} \\ \textbf{преобразований} \\
				\textbf{(БПФ, ДПФ)} \\ \textbf{(для больших размеров} \\ \textbf{сигнала)}} }
		child { node {\textbf{На основе}\\ \textbf{разложений}}
			child { node {\textbf{Общего вида}}
				child { node {\textbf{Через} \\ \textbf{быстрое} \\ \textbf{одномерное}} }
				child { node {\textbf{Через} \\\textbf{ разложение} \\ \textbf{на короткие} \\ \textbf{2D корреляции}} }  }
			child { node {\textbf{Разделяемое} \\ \textbf{ядро 2D фильтр} \\ \textbf{с разделяемым ядром}\\ \textbf{можно представить } \\ \textbf{в виде двух} \\ \textbf{1D фильтров}} } };
\end{tikzpicture}
\end{document}

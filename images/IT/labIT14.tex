\documentclass{proc}
\usepackage[utf8]{inputenc}
\usepackage[english,russian]{babel}
\usepackage{tikz}
\usepackage{float}
%\documentclass{article}
\usepackage{pgfplots}
\usetikzlibrary{snakes}
\usepackage[russian,english]{babel}
\usepackage{mathtext}
\usepgflibrary{patterns} 
\usepgflibrary[patterns] 
\usetikzlibrary{patterns} 
\usetikzlibrary[patterns]
\usepackage{amsmath,amssymb,amsfonts,textcomp,latexsym,pb-diagram,amsopn}\usepackage[utf8]{inputenc}
\pgfplotsset{compat=newest}
\begin{document}
\begin{center}
\begin{tikzpicture}

% Значения осей
\draw[dashed] (0,0.5)node[left,font=\tiny] {$200$};
\draw[dashed] (0,1)node[left,font=\tiny] {$400$};
\draw[dashed] (0,1.5)node[left,font=\tiny] {$600$};
\draw[dashed] (0,2)node[left,font=\tiny] {$800$};
\draw[dashed] (0,2.5)node[left,font=\tiny] {$1000$};
\draw[dashed] (0,0)node[left,font=\tiny] {$0$};
\draw[dashed] (2,0)node[below,font=\tiny] {$0.1$};
\draw[dashed] (5,0)node[below,font=\tiny] {$0.01$};
\draw[dashed] (8,0)node[below,font=\tiny] {$0.001$};

 \draw [pattern = crosshatch dots]
(8,0) rectangle (8.5,2.5);
\draw [fill=blue, pattern = crosshatch]
(1,0) rectangle (3,0.05);
\draw [fill=blue,  pattern = crosshatch]
(4.5,0) rectangle (5,0.05);
\draw [fill=blue, pattern = crosshatch]
(7.5,0) rectangle (8,0.05);
\draw[pattern = crosshatch dots]
(5,0) rectangle (5.5,0.3);
\draw[pattern = crosshatch dots]
(9.5,2) rectangle (10,2.5);
\node[right,scale=0.5] at (10,2.5){Метод корреляционных функций};
\draw[fill=blue,  pattern = crosshatch]
(9.5,1) rectangle (10,1.5);
\node[right,scale=0.5] at (10,1.5){Быстрый метод корреляционных функций};

\draw (0,0) -- (9,0);
\draw (0,0.5) -- (8,0.5);
\draw (0,1) -- (8,1);
\draw (0,1.5) -- (8,1.5);
\draw (0,2) -- (8,2);
\draw (0,2.5) -- (8,2.5);
\draw (8.5,0.5) -- (9,0.5);
\draw (8.5,1) -- (9,1);
\draw (8.5,1.5) -- (9,1.5);
\draw (8.5,2) -- (9,2);
\draw (8.5,2.5) -- (9,2.5);
\draw (9,2.5) -- (9,0);

\path[->] (-0.6,0) edge [white] node[above,sloped, black, scale=0.7] {Количество переборов} (-0.6,3);
\node[below,scale=0.8] at (5,-0.5){Шаг формирования эталонов};
% Оси
\draw[->] (0,0) -- (9.5,0) coordinate (x axis);
\draw[->] (0,0) -- (0,3) coordinate (y axis);
\end{tikzpicture}
\end{center}
\end{document}


{\actuality} 
Диссертационная работа посвящена совершенствованию математической модели многотонального сигнала и численных методов оценки параметров его гармоник. 

Основным направлением в применении цифровой обработки сигналов (ЦОС) является спектральный анализ. Каждый сигнал, который изменяется во времени, имеет частотный спектр. Электрические сигналы можно анализировать в частотной области с помощью анализаторов спектра, во временной области с помощью осциллографов. 

Преобразование Фурье (Fourier transform) уводит нас из временной области в частотную, и это влечет за собой огромное количество применений. Быстрое преобразование
Фурье (БПФ) представляет собой алгоритм вычисления Дискретного преобразования Фурье (ДПФ). Спектр не содержит какой-либо информации о времени, а состоит из комплексных чисел (одно число для каждой синусоиды). В комплексном числе закодированы две вещи: амплитуда и угол. 

Эффективные алгоритмы преобразования Фурье описаны в работах J.~Cooley, S.~Winograd, R.~Blahut, Г.~Нуссбаумера, Л.~М.~Гольденберга, А.~Оппенгейма и другие ученые. 

%Cooley J. W., Tukey J. W. An algorithm for the machine calculation of complex Fourier series //Mathematics of computation. – 1965. – Т. 19. – №. 90. – С. 297-301.
% Winograd S. On computing the discrete Fourier transform //Mathematics of computation. – 1978. – Т. 32. – №. 141. – С. 175-199.
% R. E. Blahut, "Algebraic fields, signal processing, and error control," in Proceedings of the IEEE, vol. 73, no. 5, pp. 874-893, May 1985, doi: 10.1109/PROC.1985.13219.
%Нуссбаумер Г. Быстрое преобразование Фурье и алгоритмы вычисления сверток. – Радио и связь, 1985.
% Гольденберг Л. М. и др. Цифровая обработка сигналов: Справочник. – Радио и связь, 1985.
% Оппенгейм А., Шафер Р. Цифровая обработка сигналов. – Litres, 2018.

Цифровой сигнал можно представить из аналогового методом дискретизации. Дискретизация сигнала может производиться по времени (выборка) и по величине сигнала (квантованием). Преобразование из аналогово сигнала в цифровой называют аналогово-цифровым преобразованием (АЦП). 

Существуют сигналы имеющие более одной независимой переменной. Такие сигналы называют одномерными. Например, сигнал изображения. Сигналы имеющие две независимые переменные называют -- двумерными.
Наиболее популярный способ сравнения таких сигналов – это их корреляция или свертка. Корреляция может быть вычислена по алгоритмам циклической свертки, для этого необходимо прочитать один из двух сигналов в обратном порядке. Класс быстрых методов вычисления корреляции состоит в ее разложении на более короткие корреляции. Z.~J.~Mou, Y.~Naito, Л.~Рабинер и другие исследователи предложили и описали эти алгоритмы в своих работах.
% Mou Z. J., Duhamel P. Fast FIR filtering: algorithms and implementations //Signal Processing. – 1987. – Т. 13. – №. 4. – С. 377-384.

% Y. Naito, T. Miyazaki and I. Kuroda, "A fast full-search motion estimation method for programmable processors with a multiply-accumulator," 1996 IEEE International Conference on Acoustics, Speech, and Signal Processing Conference Proceedings, 1996, pp. 3221-3224 vol. 6, doi: 10.1109/ICASSP.1996.550562.

%Рабинер Л., Гоулд Б. Теория и применение цифровой обработки сигналов. – Рипол Классик, 1978.

Совокупность амплитуд гармоник ряда Фурье часто называют амплитудным спектром, а совокупность фаз -- фазовым спектром. Спектр амплитуд показывает как велика составляющая каждой гармоники внутри сигнала. Гармонический сигнал определяется тремя числовыми параметрами: амплитудой, частотой и фазой. Колебания самого большого периода называют колебанием основной частоты или колебанием первой гармоники. Первая гармоника является периодической функцией с периодом $2 \pi$. Ряд Фурье четной функции содержит только косинусы. Ряд Фурье нечетной функции содержит синусы.

В ЦОС оконные функции широко используются для ограничения сигнала во времени. Классические оконные функции включают функции: Ханна, Хемминга, Блэкмана, Кайзера и другие. Оконная функция Кайзера – хорошая аппроксимация оптимального вытянутого сферического окна, концентрирующего большинство энергии в главном лепестке.
Путем корректировки параметра $\beta$ мы можем выполнить настройку окна
Кайзера.

В основе всех методов оценки параметров гармоник многотональных сигналов лежат перечисленные выше базовые алгоритмы ЦОС. Параметры гармоник находятся по спектру сигнала, получаемого с помощью БПФ, который по сути представляет собой набор корреляций с различными гармониками. В следствии ограничения по времени обрабатываемого сигнала, все методы сталкиваются с эффектами, вызванными оконными функциями.

В диссертационной работе рассмотрены известны алгоритмы, описывающие способы оценки спектральных составляющих сигнала: интерполяционные методы, метод корреляционных функций, модернизированный метод корреляционных функций и метод рекомендованный в ГОСТ, прямой корреляционный метод с использованием быстрых алгоритмов обработки сигналов. 

Значительный вклад в решение вопросов по измерению оценки параметров гармоник внесли Российские и зарубежные ученые: 
В.~Н.~Горюнов, 
% Горюнов В. Н., Осипов Д. С., Лютаревич А. Г. Определение управляющего воздействия активного фильтра гармоник //Электро. Электротехника, электроэнергетика, электротехническая промышленность. – 2009. – №. 6. – С. 20-24.
% Лютаревич, А. Г.; Горюнов, В. Н.; Долингер, С. Ю. & Хацевич, К. В. Вопросы моделирования устройств обеспечения качества электрической энергии. Омский научный вестник, Федеральное государственное бюджетное образовательное учреждение высшего ..., 2013
И.~В.~Жежеленко,
%Жежеленко И. В., Саенко Ю. Л., Бараненко Т. К. Интергармоники в системах электроснабжения промпредприятий //Вестник Приазовского государственного технического университета. Серия: Технические науки. – 1999. – №. 8.
% Жежеленко И. В. Высшие гармоники в системах электроснабжения промпредприятий. – 2010.
В.~Т.~Черемисин,
% Черемисин В. Т., Дубовик Е. П. Способ расчета высших гармоник, генерируемых несколькими электротяговыми нагрузками //Динамика электрических машин. – 1985. – С. 150-153.
% Черемисин В. Т., Грицутенко С. С. Способ повышения точности измерения гармонических составляющих тягового тока и напряжения //Вестник Ростовского государственного университета путей сообщения. – 2007. – №. 2. – С. 94-99.
E.~Jacobsen, 
% E. Jacobsen and P. Kootsookos, "Fast, Accurate Frequency Estimators [DSP Tips & Tricks]," in IEEE Signal Processing Magazine, vol. 24, no. 3, pp. 123-125, May 2007, doi: 10.1109/MSP.2007.361611.
%Jacobsen E. On local interpolation of DFT outputs //Available online: http://www. ericjacobsen. org/FTinterp. pdf.[Accessed January 2013]. – 1994.
% Jacobsen, E. & Kootsookos, P. Fast, accurate frequency estimators [DSP Tips & Tricks] IEEE Signal Processing Magazine, IEEE, 2007, 24, 123-125
M.~D.~Macleod,
% Macleod, M. Fast nearly ML estimation of the parameters of real or complex single tones or resolved multiple tones IEEE Transactions on Signal Processing, 1998, 46, 141-148
% Rife, D. & Boorstyn, R. Single tone parameter estimation from discrete-time observations IEEE Transactions on Information Theory, 1974, 20, 591-598
B.~G.~Quinn,
% Quinn, B. Estimating frequency by interpolation using Fourier coefficients IEEE Transactions on Signal Processing, 1994, 42, 1264-1268
% Quinn, B. & Kootsookos, P. Threshold behavior of the maximum likelihood estimator of frequency IEEE Transactions on Signal Processing, 1994, 42, 3291-3294
% Quinn, B. G. Estimation of frequency, amplitude, and phase from the DFT of a time series IEEE Transactions on Signal Processing, 1997, 45, 814-817
Thomas~Grandke,
% Grandke, T. Interpolation Algorithms for Discrete Fourier Transforms of Weighted Signals IEEE Transactions on Instrumentation and Measurement, 1983, 32, 350-355
Cheng-I~Chen,
%C. Chen and Y. Chen, "Comparative Study of Harmonic and Interharmonic Estimation Methods for Stationary and Time-Varying Signals," in IEEE Transactions on Industrial Electronics, vol. 61, no. 1, pp. 397-404, Jan. 2014, doi: 10.1109/TIE.2013.2242419.
Jiufei~Luo, Zhijiang~Xie, Ming~Xie,
% Luo, J. & Xie, M. Phase difference methods based on asymmetric windows. Mechanical Systems and Signal Processing, Elsevier, 2015, 54, 52-67
% Luo J., Xie Z., Xie M. Interpolated DFT algorithms with zero padding for classic windows //Mechanical Systems and Signal Processing. – 2016. – Т. 70. – С. 1011-1025.
Krzysztof~Duda,
% Duda, K. DFT interpolation algorithm for Kaiser--Bessel and Dolph--Chebyshev windows. IEEE Transactions on Instrumentation and Measurement, IEEE, 2011, 60, 784-790
Shen Zhou, Zhang Shancong,
% Zhou S., Shancong Z. Improved frequency estimation algorithm by least squares phase unwrapping //Circuits, Systems, and Signal Processing. – 2018. – Т. 37. – №. 12. – С. 5680-5687.
Carlos~Joao~Ramos, Antonio~Pina~Martins, Adriano~da~Silva~Carvalho и другие ученые.
% Ramos, C. J.; Martins, A. P. & da Silva Carvalho, A. Power system frequency estimation using a least mean squares differentiator International Journal of Electrical Power & Energy Systems, Elsevier, 2017, 87, 166-175
% Ramos, C.; Martins, A. & Carvalho, A. Synchronizing renewable energy sources in distributed generation systems Proc. Int. Conf. Renew. Energy Power Qual.(ICREPQ), Zaragoza, Spain, 2005, 1-5
% Ramos, C. J.; Martins, A. P. & Carvalho, A. S. Frequency and Phase-Angle Estimation Using Ordinary Least Squares IEEE Transactions on Industrial Electronics, 2015, 62, 5677-5688

{\aim} данной работы является совершенствование алгоритмов оценки параметров гармоник многотонального сигнала.

%{\researchObject} являются алгоритмы оценки параметров гармоник в силовых электрических сетях.

%{\researchSubject} является точность и быстродействие алгоритмов оценки параметров гармоник.

Для~достижения поставленной цели необходимо было решить следующие {\tasks}:
\begin{enumerate}
  \item Анализ математических основ объекта исследования и формулировка математической модели многотонального сигнала.
  
  \item Изучение и экспериментальное исследование алгоритмов оценки параметров гармоник.
  
  \item Развитие математической модели многотональных сигналов в части расчета точности оценки амплитуды применительно к используемому при оценке параметров гармоник подходу, связанному с применением оконных функций.
  
  \item Разработка численных методов для оценки параметров гармоник, позволяющих достичь расчетной точности для амплитуды гармоники.
  
  \item Разработка алгоритмов для эффективного выполнения численных методов из предыдущей задачи.
  
  \item Разработка комплекса программа для анализа и доработки алгоритмов оценки параметров гармоник многотональных сигналов.
\end{enumerate}


{\novelty}
%В частности, ВАК отклонила диссертацию, в которой автор «представил научную новизну в виде  процесса,  а  не  результата».
% В формулировке положений,  выносимых  на защиту, должны  содержаться  отличительные признаки новых  научных результатов,  характеризующие  вклад  соискателя  в область науки,  к которой относится тема диссертации. Они должны содержать не только краткое изложение сущности  полученных  результатов,  но  и  сравнительную  оценку  их  научной  и  практической значимости.
 
\begin{enumerate}
  \item Уточнена математическая модель спектра многотонального сигнала полученными и экспериментально проверенными формулами для нахождения границы Крамера-Рао при оценке амплитуды гармоники для взвешенного оконной функцией сигнала.
  
  \item Предложен численный метод нахождения оптимальной несмещенной оценки амплитуды гармоники на основе корреляционного анализа, а также предложена его быстрая реализация на основе алгоритмов разряженного БПФ.
  
  \item Реализован комплекс программ для экспериментальной проверки полученных в работе формул и анализа алгоритмов оценки параметров гармоник многотональных сигналов.
\end{enumerate}

{\influence} 

\begin{enumerate}
	\item Выведена формула для нахождения границы Крамера-Рао при применении оконной функции, которая позволяет повысить эффективность научных исследований различных алгоритмов обработки сигналов с применением оконных функций, заменив моделирование алгоритма с применением различных окон расчетом по предложенной формуле.
	
	\item Предложенный численный метод, вместе с его быстрой реализацией, позволяют повысить точность и достоверность результатов измерительных приборов для электрических сетей.
	
	\item Разработанный комплекс программ позволяет проводить научные исследования в области цифровой обработки сигналов и используется в учебном процессе.
\end{enumerate}

На основании теоретических и экспериментальных исследований разработан и зарегистрирована программа, написанная на языке программирования Python, позволяющая решать задачи спектрального анализа напряжений в электроэнергетической системе, а так же оценивать параметры гармоник многотонального сигнала.


{\methods} 
Исследование состоит в итерационном изучении объекта исследования с попеременным уточнением его модели с точки зрения математического описания и с точки зрения его программной реализации. При этом использовались методы исследования основанные на математическом анализе и математической статистики с одной стороны и методы численного моделирования с использованием платформы SciPy для языка программирования Python с другой.

% Второй вариант методов
%Для решения поставленных задач в диссертационной работе использовались методы математического анализа и теории вероятностей, численные методы. Исследования алгоритмов осуществлялось на ЭВМ с помощью языков программирования Python и С. 


{\defpositions}
\begin{enumerate}
  \item Дополнение к математической модели многотонального сигнала в виде формулы, позволяющей определить дисперсию оценки амплитуды гармоники, отличающемуся от известной границы Крамера-Рао учетом изменения дисперсии после применения оконных функций.
  \item Основанный на корреляционном анализе численный метод, позволяющий определить параметры гармоник сигналов с точностью, определяемой уточненной границей Крамера-Рао, включающий в себя вычислительно-эффективную схему расчета корреляций и отличающийся от известных методов отсутствием потерь в точности результатов при интерполировании параметров гармоник.
  \item Комплекс программ для анализа и построения алгоритмов оценки параметров многотональных сигналов.
\end{enumerate}
%В папке Documents можно ознакомиться в решением совета из Томского ГУ
%в~файле \verb+Def_positions.pdf+, где обоснованно даются рекомендации
%по~формулировкам защищаемых положений.

{\reliability} 
научных результатов, выводов и рекомендаций диссертации определяется корректным применением
общенаучных методов исследования и математических методов, а также
надежной информационной базой исследования. Выводы диссертационного исследования не противоречат известным теоретическим и практическим результатам, содержащимся в трудах отечественных и зарубежных ученых в области улучшения алгоритмов оценки параметров многотонального сигнала.


% второй вариант
%полученных результатов обеспечивается теоретически и на практике при проведении эксперимента, подтверждена исследованиями на ЭВМ, написано на языке программирования Python, в среде IDE JetBrains PyCharm Community Edition 2018.3.1 x64.  

{\probation}
Основные результаты работы докладывались~на:

\begin{enumerate}

\item Научная конференция, посвященная Дню Российской науки <<Инновационные проекты и технологии в образовании, промышленности и на транспорте>> (Омск, 2016) \cite{detectors_local2016}.	

\item  Всероссийская научно-практическая конференция <<Информационно-телекоммуникационная системы и технологии>>
(Кемерово, 2016) \cite{boundary_recognition2017}.

\item XI научная конференция, посвященной Дню российской науки <<Инновационные проекты и технологии в образовании, промышленности и на транспорте>> (Омск, 2017) \cite{innovative_projects2017}

\item V Всероссийская научно-техническая конференция с международным участием <<Технологическое обеспечение ремонта и повышение динамических качеств железнодорожного подвижного состава>> (Омск, 2017)  \cite{implementation_of_algorithms2017}

\item II Международная научно-практическая конференция  <<САПР и моделирование в современной электронике>> (Брянск, 2018)  \cite{information-measuring2018}

\item XIX Всероссийская конференция молодых ученых по математическому моделированию и информационным технологиям <<Тезисы XIX всероссийской конференции молодых ученых  по математическому моделированию и информационным технологиям>> (Кемерово, 2018)  \cite{efficiency_mark2018}.

\item II Всероссийская научно-техническая конференция с международным участием <<Информационные и управляющие системы на транспорте и в промышленности>> (Омск, 2018) \cite{accuracy_study2018}.

\item Научная конференция, посвященная Дню Российской науки <<Инновационные проекты и технологии в образовании, промышленности и на транспорте>> (Омск, 2018) \cite{depth_expansion2018}.

\item XXI Вcероссийская научно-техническая конференция <<Современные проблемы радоэлектроники>> (Красноярск, 2018) \cite{application_of_fast2018}.

\item III Международная научно-техническая конференция <<Проблемы машиноведения>> (Омск, 2019) \cite{cache_oriented2019}.

\item I Всероссийская научно-практическая конференция с международным участием <<Системы управления, информационные технологии и математическое моделирование>> (Омск, 2019) \cite{comparative_analysis_2019}.

\item Материалы научной конференции, посвященной Дню Российской науки <<Инновационные проекты и технологии в образовании, промышленности и на транспорте>> (Омск, 2019) \cite{comparative_analysis2019}.

\item Материалы всероссийской научно-технической конференции <<Надежность функционирования и информационная безопасность инфокоммуникационных, телекоммуникационных и радиотехнических сетей и систем>> (Омск, 2019) \cite{modern_information2019}. 

\item III Международная научно-практическая конференция <<Инновации в информационных технологиях, машиностроении и автотранспорте>> \cite{complexity_assessment2019}.

\item Материалы научной конференции, посвященной Дню Российской науки <<Инновационные проекты и технологии в образовании, промышленности и на транспорте>> (Омск, 2020) \cite{comparative_study2020}.  

\item Материалы научной конференции, посвященной Дню Российской науки <<Инновационные проекты и технологии в образовании, промышленности и на транспорте>> (Омск, 2021) \cite{altman2021boundary}
\end{enumerate}
%{\contribution} Автор принимал активное участие \ldots

\ifnumequal{\value{bibliosel}}{0}
{%%% Встроенная реализация с загрузкой файла через движок bibtex8. (При желании, внутри можно использовать обычные ссылки, наподобие `\cite{bib1,vakbib2}`).
    {\publications} Основные результаты по теме диссертации изложены в XX печатных изданиях,
    X из которых изданы в журналах, рекомендованных ВАК,
    X "--- в тезисах докладов.
}%
{%%% Реализация пакетом biblatex через движок biber
    \begin{refsection}[bl-author]
        % Это refsection=1.
        % Процитированные здесь работы:
        %  * подсчитываются, для автоматического составления фразы "Основные результаты ..."
        %  * попадают в авторскую библиографию, при usefootcite==0 и стиле `\insertbiblioauthor` или `\insertbiblioauthorgrouped`
        %  * нумеруются там в зависимости от порядка команд `\printbibliography` в этом разделе.
        %  * при использовании `\insertbiblioauthorgrouped`, порядок команд `\printbibliography` в нём должен быть тем же (см. biblio/biblatex.tex)
        %
        % Невидимый библиографический список для подсчёта количества публикаций:
        \printbibliography[heading=nobibheading, section=1, env=countauthorvak,          keyword=biblioauthorvak]%
        \printbibliography[heading=nobibheading, section=1, env=countauthorwos,          keyword=biblioauthorwos]%
        \printbibliography[heading=nobibheading, section=1, env=countauthorscopus,       keyword=biblioauthorscopus]%
        \printbibliography[heading=nobibheading, section=1, env=countauthorconf,         keyword=biblioauthorconf]%
        \printbibliography[heading=nobibheading, section=1, env=countauthorother,        keyword=biblioauthorother]%
        \printbibliography[heading=nobibheading, section=1, env=countauthor,             keyword=biblioauthor]%
        \printbibliography[heading=nobibheading, section=1, env=countauthorvakscopuswos, filter=vakscopuswos]%
        \printbibliography[heading=nobibheading, section=1, env=countauthorscopuswos,    filter=scopuswos]%
        %
        \nocite{*}%
        %
        {\publications} Основные результаты по теме диссертации изложены в~\arabic{citeauthor}~печатных изданиях,
        \arabic{citeauthorvak} из которых изданы в журналах, рекомендованных ВАК\sloppy%
        \ifnum \value{citeauthorscopuswos}>0%
            , \arabic{citeauthorscopuswos} "--- в~периодических научных журналах, индексируемых Web of~Science и Scopus\sloppy%
        \fi%
        \ifnum \value{citeauthorconf}>0%
            , \arabic{citeauthorconf} "--- в~тезисах докладов.
        \else%
            .
        \fi
    \end{refsection}%
    \begin{refsection}[bl-author]
        % Это refsection=2.
        % Процитированные здесь работы:
        %  * попадают в авторскую библиографию, при usefootcite==0 и стиле `\insertbiblioauthorimportant`.
        %  * ни на что не влияют в противном случае
%        \nocite{vakbib2}%vak
%        \nocite{bib1}%other
%        \nocite{confbib1}%conf
    \end{refsection}%
        %
        % Всё, что вне этих двух refsection, это refsection=0,
        %  * для диссертации - это нормальные ссылки, попадающие в обычную библиографию
        %  * для автореферата:
        %     * при usefootcite==0, ссылка корректно сработает только для источника из `external.bib`. Для своих работ --- напечатает "[0]" (и даже Warning не вылезет).
        %     * при usefootcite==1, ссылка сработает нормально. В авторской библиографии будут только процитированные в refsection=0 работы.
        %
        % Невидимый библиографический список для подсчёта количества внешних публикаций
        % Используется, чтобы убрать приставку "А" у работ автора, если в автореферате нет
        % цитирований внешних источников.
        % Замедляет компиляцию
    \ifsynopsis
    \ifnumequal{\value{draft}}{0}{
      \printbibliography[heading=nobibheading, section=0, env=countexternal,          keyword=biblioexternal]%
    }{}
    \fi
}

%При использовании пакета \verb!biblatex! будут подсчитаны все работы, добавленные
%в файл \verb!biblio/author.bib!. Для правильного подсчёта работ в~различных
%системах цитирования требуется использовать поля:
%\begin{itemize}
%        \item \texttt{authorvak} если публикация индексирована ВАК,
%        \item \texttt{authorscopus} если публикация индексирована Scopus,
%        \item \texttt{authorwos} если публикация индексирована Web of Science,
%        \item \texttt{authorconf} для докладов конференций,
%        \item \texttt{authorother} для других публикаций.
%\end{itemize}
%
%
%
%
%Для подсчёта используются счётчики:
%\begin{itemize}
%        \item \texttt{citeauthorvak} для работ, индексируемых ВАК,
%        \item \texttt{citeauthorscopus} для работ, индексируемых Scopus,
%        \item \texttt{citeauthorwos} для работ, индексируемых Web of Science,
%        \item \texttt{citeauthorvakscopuswos} для работ, индексируемых одной из трёх баз,
%        \item \texttt{citeauthorscopuswos} для работ, индексируемых Scopus или Web of~Science,
%        \item \texttt{citeauthorconf} для докладов на конференциях,
%        \item \texttt{citeauthorother} для остальных работ,
%        \item \texttt{citeauthor} для суммарного количества работ.
%\end{itemize}
%% Счётчик \texttt{citeexternal} используется для подсчёта процитированных публикаций.
%
%Для добавления в список публикаций автора работ, которые не были процитированы в
%автореферате требуется их~перечислить с использованием команды \verb!\nocite! в
%\verb!Synopsis/content.tex!.

{\actuality} 
Диссертационная работа посвящена совершенствованию математической модели многотонального сигнала и численных методов оценки параметров его гармоник. 

Основным направлением в применении цифровой обработки сигналов (ЦОС) является спектральный анализ. Каждый сигнал, который изменяется во времени, имеет частотный спектр. Электрические сигналы можно анализировать в частотной области с помощью анализаторов спектра, во временной области с помощью осциллографов. 

Преобразование Фурье уводит нас из временной области в частотную, и это
влечет за собой огромное количество применений. Быстрое преобразование
Фурье (БПФ) представляет собой алгоритм вычисления Дискретного преобразования Фурье (ДПФ). Спектр не содержит какой-либо информации о времени, а состоит из комплексных чисел (одно число для каждой синусоиды). В комплексном числе закодированы две вещи: амплитуда и угол. Спектр может содержать очень большие и  очень малые значения. Если взять логарифм этого спектра, то этот диапазон значений будет существенно сжат.

Процесс наложения окна на сигнал, представляет собой умножение неограниченной во времени функции на ограниченную весовую функцию окна. Основная цель наложения окна – получение ограниченной во времени функции, при минимальных спектральных искажениях.

Классические оконные функции включают функции Ханна, Хемминга и Блэкмана.
Оконная функция Кайзера – хорошая аппроксимация оптимального вытянутого сферического окна, концентрирующего большинство энергии в главном лепестке.
Путем корректировки параметра $\beta$ мы можем выполнить настройку окна
Кайзера.

В диссертационной работе рассмотрены известны алгоритмы, описывающие способы оценки спектральных составляющих сигнала: интерполяционные методы, метод корреляционных функций, модернизированный метод корреляционных функций и метод рекомендованный в ГОСТ, прямой корреляционный метод с использованием быстрых алгоритмов обработки сигналов. 

Эффективные алгоритмы преобразования Фурье описаны в работах J.~Cooley, S.~Winograd, R.~Blahut, Г.~Нуссбаумера, Л.~М.~Гольденберга, А.~Оппенгейма и другие ученые. Класс быстрых методов вычисления корреляции состоит в ее разложении на более короткие корреляции. Z.~J.~Mou, Y.~Naito, Л.~Рабинер и другие исследователи предложили и описали эти алгоритмы в своих работах \cite{MOU1987377, 550562, Rabiner1978theory}.

Значительный вклад в решение вопросов по измерению оценки параметров гармоник внесли зарубежные ученые: Eric~Jacobsen's \cite{4205098, jacobsen1994local, jacobsen2007fast}, B.~G.~Quinn \cite{295186, 330402, 558515}, Cheng-I Chen, Shen Zhou, Zhang Shancong \cite{zhou2018improved}, Krzysztof Duda,  Dnyaneshwar D. Ahire, Javad Enayati, Zahra Moravej, Magnago Fernando, Reineri Claudio and Lovera Santiago, Zhengguang Xu, Tianqi Lu, and Benxiong Huang.      

% B. G. Quinn, "Estimating frequency by interpolation using Fourier coefficients," in IEEE Transactions on Signal Processing, vol. 42, no. 5, pp. 1264-1268, May 1994, doi: 10.1109/78.295186.
% B. G. Quinn and P. J. Kootsookos, "Threshold behavior of the maximum likelihood estimator of frequency," in IEEE Transactions on Signal Processing, vol. 42, no. 11, pp. 3291-3294, Nov. 1994, doi: 10.1109/78.330402.
% B. G. Quinn, "Estimation of frequency, amplitude, and phase from the DFT of a time series," in IEEE Transactions on Signal Processing, vol. 45, no. 3, pp. 814-817, March 1997, doi: 10.1109/78.558515.

% C. Chen and Y. Chen, "Comparative Study of Harmonic and Interharmonic Estimation Methods for Stationary and Time-Varying Signals," in IEEE Transactions on Industrial Electronics, vol. 61, no. 1, pp. 397-404, Jan. 2014, doi: 10.1109/TIE.2013.2242419.

% Zhou S., Shancong Z. Improved frequency estimation algorithm by least squares phase unwrapping //Circuits, Systems, and Signal Processing. – 2018. – Т. 37. – №. 12. – С. 5680-5687.

{\aim} данной работы является совершенствование алгоритмов оценки параметров гармоник многотонального сигнала.

{\researchObject} являются алгоритмы оценки параметров гармоник в силовых электрических сетях.

{\researchSubject} является точность и быстродействие алгоритмов оценки параметров гармоник.

Для~достижения поставленной цели необходимо было решить следующие {\tasks}:
\begin{enumerate}
  \item Анализ математических основ объекта исследования и формулировка математической модели многотонального сигнала.
  
  \item Изучение и экспериментальное исследование алгоритмов оценки параметров гармоник.
  
  \item Развитие математической модели многотональных сигналов в части расчета точности оценки амплитуды применительно к используемому при оценке параметров гармоник подходу, связанному с применением оконных функций.
  
  \item Разработка численных методов для оценки параметров гармоник, позволяющих достичь расчетной точности для амплитуды гармоники.
  
  \item Разработка алгоритмов для эффективного выполнения численных методов из предыдущей задачи.
  
  \item Разработка комплекса программа для анализа и доработки алгоритмов оценки параметров гармоник многотональных сигналов.
\end{enumerate}


{\novelty}
%В частности, ВАК отклонила диссертацию, в которой автор «представил научную новизну в виде  процесса,  а  не  результата».
% В формулировке положений,  выносимых  на защиту, должны  содержаться  отличительные признаки новых  научных результатов,  характеризующие  вклад  соискателя  в область науки,  к которой относится тема диссертации. Они должны содержать не только краткое изложение сущности  полученных  результатов,  но  и  сравнительную  оценку  их  научной  и  практической значимости.
 
\begin{enumerate}
  \item Уточнена математическая модель спектра многотонального сигнала полученными и экспериментально проверенными формулами для нахождения границы Крамера-Рао при оценке амплитуды гармоники для взвешенного оконной функцией сигнала.
  
  \item Предложен численный метод нахождения оптимальной несмещенной оценки амплитуды гармоники на основе корреляционного анализа, а также предложена его быстрая реализация на основе алгоритмов разряженного БПФ.
  
  \item Реализован комплекс программ для экспериментальной проверки полученных в работе формул и анализа алгоритмов оценки параметров гармоник многотональных сигналов.
\end{enumerate}

{\influence} 

\begin{enumerate}
	\item Выведена формула для нахождения границы Крамера-Рао при применении оконной функции, которая позволяет повысить эффективность научных исследований различных алгоритмов обработки сигналов с применением оконных функций, заменив моделирование алгоритма с применением различных окон расчетом по предложенной формуле.
	
	\item Предложенный численный метод, вместе с его быстрой реализацией, позволяют повысить точность и достоверность результатов измерительных приборов для электрических сетей.
	
	\item Разработанный комплекс программ позволяет проводить научные исследования в области цифровой обработки сигналов и используется в учебном процессе.
\end{enumerate}

На основании теоретических и экспериментальных исследований разработан и зарегистрирована программа, написанная на языке программирования Pytnon, позволяющая решать задачи спектрального анализа напряжений в электроэнергетической системе, а так же оценивать параметры гармоник многотонального сигнала.


{\methods} 
Исследования состояла в итерационном изучении объекта исследования с попеременным уточнением его модели с точки зрения математического описания и с точки зрения его программной реализации. При этом использовались методы исследования основанные на математическом анализе и математической статистики с одной стороны и методы численного моделирования с использованием платформы SciPy для языка программирования Python с другой.

% Второй вариант методов
%Для решения поставленных задач в диссертационной работе использовались методы математического анализа и теории вероятностей, численные методы. Исследования алгоритмов осуществлялось на ЭВМ с помощью языков программирования Python и С. 


{\defpositions}
\begin{enumerate}
  \item Дополнение к математической модели многотонального сигнала в виде формулы, позволяющей определить дисперсию оценки амплитуды гармоники, отличающемуся от известной границы Крамера-Рао учетом изменения дисперсии после применения оконных функций.
  \item Основанный на корреляционном анализе численный метод, позволяющий определить параметры гармоник сигналов с точностью, определяемой уточненной границей Крамера-Рао, включающий в себя вычислительно-эффективную схему расчета корреляций и отличающийся от известных методов отсутствием потерь в точности результатов при интерполировании параметров гармоник.
  \item Комплекс программ для анализа и построения алгоритмов оценки параметров многотональных сигналов.
\end{enumerate}
%В папке Documents можно ознакомиться в решением совета из Томского ГУ
%в~файле \verb+Def_positions.pdf+, где обоснованно даются рекомендации
%по~формулировкам защищаемых положений.

{\reliability} 
научных результатов, выводов и рекомендаций диссертации определяются корректным применением
общенаучных методов исследования и математических методов, а также
надежной информационной базой исследования. Выводы диссертационного исследования не противоречат известным теоретическим и практическим результатам, содержащимся в трудах отечественных и зарубежных ученых в области повышения энергетической эффективности.


% второй вариант
%полученных результатов обеспечивается теоретически и на практике при проведении эксперимента, подтверждена исследованиями на ЭВМ, написано на языке программирования Python, в среде IDE JetBrains PyCharm Community Edition 2018.3.1 x64.  

{\probation}
Основные результаты работы докладывались~на:
\begin{enumerate}
\item Инновационные проекты и технологии в образовании, промышленности и на транспорте \cite{comparative_study2020}.  

\item Инновации в информационных технологиях, машиностроении и автотранспорте. Сборник материалов III Международной научно-практической конференции \cite{complexity_assessment2019}.

\item Надежность функционирования и информационная безопасность инфокоммуникационных, телекоммуникационных и радиотехнических сетей и систем. Материалы всероссийской научно-технической конференции. \cite{modern_information2019}. 

\item Инновационные проекты и технологии в образовании, промышленности и на транспорте. Материалы научной конференции, посвященной Дню Российской науки. Омский государственный университет путей сообщения. \cite{comparative_analysis2019}.

\item Системы управления, информационные технологии и математическое моделирование. Материалы I Всероссийской научно-практической конференции с международным участием \cite{comparative_analysis_2019}.

\item Проблемы машиноведения. Материалы III Международной научно-технической конференции \cite{cache_oriented2019}.

\item Информационные и управляющие системы на транспорте и в промышленности. Материалы II всероссийской научно-технической конференции \cite{accuracy_study2018}.

\item Тезисы XIX Всероссийской конференции молодых учёных по математическому моделированию и информационным технологиям. Тезисы докладов \cite{efficiency_mark2018}.

\item  САПР и моделирование в современной электронике. Cборник научных трудов II Международной научно-практической конференции \cite{information-measuring2018}

\end{enumerate}
%{\contribution} Автор принимал активное участие \ldots

\ifnumequal{\value{bibliosel}}{0}
{%%% Встроенная реализация с загрузкой файла через движок bibtex8. (При желании, внутри можно использовать обычные ссылки, наподобие `\cite{bib1,vakbib2}`).
    {\publications} Основные результаты по теме диссертации изложены в XX печатных изданиях,
    X из которых изданы в журналах, рекомендованных ВАК,
    X "--- в тезисах докладов.
}%
{%%% Реализация пакетом biblatex через движок biber
    \begin{refsection}[bl-author]
        % Это refsection=1.
        % Процитированные здесь работы:
        %  * подсчитываются, для автоматического составления фразы "Основные результаты ..."
        %  * попадают в авторскую библиографию, при usefootcite==0 и стиле `\insertbiblioauthor` или `\insertbiblioauthorgrouped`
        %  * нумеруются там в зависимости от порядка команд `\printbibliography` в этом разделе.
        %  * при использовании `\insertbiblioauthorgrouped`, порядок команд `\printbibliography` в нём должен быть тем же (см. biblio/biblatex.tex)
        %
        % Невидимый библиографический список для подсчёта количества публикаций:
        \printbibliography[heading=nobibheading, section=1, env=countauthorvak,          keyword=biblioauthorvak]%
        \printbibliography[heading=nobibheading, section=1, env=countauthorwos,          keyword=biblioauthorwos]%
        \printbibliography[heading=nobibheading, section=1, env=countauthorscopus,       keyword=biblioauthorscopus]%
        \printbibliography[heading=nobibheading, section=1, env=countauthorconf,         keyword=biblioauthorconf]%
        \printbibliography[heading=nobibheading, section=1, env=countauthorother,        keyword=biblioauthorother]%
        \printbibliography[heading=nobibheading, section=1, env=countauthor,             keyword=biblioauthor]%
        \printbibliography[heading=nobibheading, section=1, env=countauthorvakscopuswos, filter=vakscopuswos]%
        \printbibliography[heading=nobibheading, section=1, env=countauthorscopuswos,    filter=scopuswos]%
        %
        \nocite{*}%
        %
        {\publications} Основные результаты по теме диссертации изложены в~\arabic{citeauthor}~печатных изданиях,
        \arabic{citeauthorvak} из которых изданы в журналах, рекомендованных ВАК\sloppy%
        \ifnum \value{citeauthorscopuswos}>0%
            , \arabic{citeauthorscopuswos} "--- в~периодических научных журналах, индексируемых Web of~Science и Scopus\sloppy%
        \fi%
        \ifnum \value{citeauthorconf}>0%
            , \arabic{citeauthorconf} "--- в~тезисах докладов.
        \else%
            .
        \fi
    \end{refsection}%
    \begin{refsection}[bl-author]
        % Это refsection=2.
        % Процитированные здесь работы:
        %  * попадают в авторскую библиографию, при usefootcite==0 и стиле `\insertbiblioauthorimportant`.
        %  * ни на что не влияют в противном случае
%        \nocite{vakbib2}%vak
%        \nocite{bib1}%other
%        \nocite{confbib1}%conf
    \end{refsection}%
        %
        % Всё, что вне этих двух refsection, это refsection=0,
        %  * для диссертации - это нормальные ссылки, попадающие в обычную библиографию
        %  * для автореферата:
        %     * при usefootcite==0, ссылка корректно сработает только для источника из `external.bib`. Для своих работ --- напечатает "[0]" (и даже Warning не вылезет).
        %     * при usefootcite==1, ссылка сработает нормально. В авторской библиографии будут только процитированные в refsection=0 работы.
        %
        % Невидимый библиографический список для подсчёта количества внешних публикаций
        % Используется, чтобы убрать приставку "А" у работ автора, если в автореферате нет
        % цитирований внешних источников.
        % Замедляет компиляцию
    \ifsynopsis
    \ifnumequal{\value{draft}}{0}{
      \printbibliography[heading=nobibheading, section=0, env=countexternal,          keyword=biblioexternal]%
    }{}
    \fi
}

%При использовании пакета \verb!biblatex! будут подсчитаны все работы, добавленные
%в файл \verb!biblio/author.bib!. Для правильного подсчёта работ в~различных
%системах цитирования требуется использовать поля:
%\begin{itemize}
%        \item \texttt{authorvak} если публикация индексирована ВАК,
%        \item \texttt{authorscopus} если публикация индексирована Scopus,
%        \item \texttt{authorwos} если публикация индексирована Web of Science,
%        \item \texttt{authorconf} для докладов конференций,
%        \item \texttt{authorother} для других публикаций.
%\end{itemize}
%
%
%
%
%Для подсчёта используются счётчики:
%\begin{itemize}
%        \item \texttt{citeauthorvak} для работ, индексируемых ВАК,
%        \item \texttt{citeauthorscopus} для работ, индексируемых Scopus,
%        \item \texttt{citeauthorwos} для работ, индексируемых Web of Science,
%        \item \texttt{citeauthorvakscopuswos} для работ, индексируемых одной из трёх баз,
%        \item \texttt{citeauthorscopuswos} для работ, индексируемых Scopus или Web of~Science,
%        \item \texttt{citeauthorconf} для докладов на конференциях,
%        \item \texttt{citeauthorother} для остальных работ,
%        \item \texttt{citeauthor} для суммарного количества работ.
%\end{itemize}
%% Счётчик \texttt{citeexternal} используется для подсчёта процитированных публикаций.
%
%Для добавления в список публикаций автора работ, которые не были процитированы в
%автореферате требуется их~перечислить с использованием команды \verb!\nocite! в
%\verb!Synopsis/content.tex!.

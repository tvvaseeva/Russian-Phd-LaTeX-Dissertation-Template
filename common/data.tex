%%% Основные сведения %%%
\newcommand{\thesisAuthorLastName}{Васеева}
\newcommand{\thesisAuthorOtherNames}{Татьяна Валериевна}
\newcommand{\thesisAuthorInitials}{Т.\,В.}
\newcommand{\thesisAuthor}             % Диссертация, ФИО автора
{%
    \texorpdfstring{% \texorpdfstring takes two arguments and uses the first for (La)TeX and the second for pdf
        \thesisAuthorLastName~\thesisAuthorOtherNames% так будет отображаться на титульном листе или в тексте, где будет использоваться переменная
    }{%
        \thesisAuthorLastName, \thesisAuthorOtherNames% эта запись для свойств pdf-файла. В таком виде, если pdf будет обработан программами для сбора библиографических сведений, будет правильно представлена фамилия.
    }
}
\newcommand{\thesisAuthorShort}        % Диссертация, ФИО автора инициалами
{\thesisAuthorInitials~\thesisAuthorLastName}
%\newcommand{\thesisUdk}                % Диссертация, УДК
%{\todo{xxx.xxx}}
\newcommand{\thesisTitle}              % Диссертация, название
{\textsc{совершенствование математической модели многотонального сигнала и численных методов оценки параметров его гармоник}}
\newcommand{\thesisSpecialtyNumber}    % Диссертация, специальность, номер
{05.13.18}
\newcommand{\thesisSpecialtyTitle}     % Диссертация, специальность, название (название взято с сайта ВАК для примера)
{Математическое моделирование, численные методы и комплексы программ}
%% \newcommand{\thesisSpecialtyTwoNumber} % Диссертация, вторая специальность, номер
%% {\todo{XX.XX.XX}}
%% \newcommand{\thesisSpecialtyTwoTitle}  % Диссертация, вторая специальность, название
%% {\todo{Теория и~методика физического воспитания, спортивной тренировки,
%% оздоровительной и~адаптивной физической культуры}}
\newcommand{\thesisDegree}             % Диссертация, ученая степень
{кандидата технических наук}
\newcommand{\thesisDegreeShort}        % Диссертация, ученая степень, краткая запись
{канд. техн. наук}
\newcommand{\thesisCity}               % Диссертация, город написания диссертации
{Омск}
\newcommand{\thesisYear}               % Диссертация, год написания диссертации
{2021}
\newcommand{\thesisOrganization}       % Диссертация, организация
{Федеральное государственное бюджетное образовательное учреждение высшего
образования <<Омский государственный университет путей сообщения>> (ОмГУПС)}
\newcommand{\thesisOrganizationShort}  % Диссертация, краткое название организации для доклада
{Оценка~параметров~гармоник}

\newcommand{\thesisInOrganization}     % Диссертация, организация в предложном падеже: Работа выполнена в ...
{<<федеральном государственном бюджетном образовательном учреждении высшего профессионального образования «Омский государственный университет путей сообщения (ОмГУПС (ОмИИТ)>>}

%% \newcommand{\supervisorDead}{}           % Рисовать рамку вокруг фамилии
\newcommand{\supervisorFio}              % Научный руководитель, ФИО
{Альтман Евгений Анатольевич}
\newcommand{\supervisorRegalia}          % Научный руководитель, регалии
{кандидат технических наук, доцент}
\newcommand{\supervisorFioShort}         % Научный руководитель, ФИО
{Е.\,А.~Альтман}
\newcommand{\supervisorRegaliaShort}     % Научный руководитель, регалии
{кандидат технических наук, доцент}

%% \newcommand{\supervisorTwoDead}{}        % Рисовать рамку вокруг фамилии
%% \newcommand{\supervisorTwoFio}           % Второй научный руководитель, ФИО
%% {\todo{Фамилия Имя Отчество}}
%% \newcommand{\supervisorTwoRegalia}       % Второй научный руководитель, регалии
%% {\todo{уч. степень, уч. звание}}
%% \newcommand{\supervisorTwoFioShort}      % Второй научный руководитель, ФИО
%% {\todo{И.\,О.~Фамилия}}
%% \newcommand{\supervisorTwoRegaliaShort}  % Второй научный руководитель, регалии
%% {\todo{уч.~ст.,~уч.~зв.}}

\newcommand{\opponentOneFio}           % Оппонент 1, ФИО
{\todo{Фамилия Имя Отчество}}
\newcommand{\opponentOneRegalia}       % Оппонент 1, регалии
{\todo{доктор физико-математических наук, профессор}}
\newcommand{\opponentOneJobPlace}      % Оппонент 1, место работы
{\todo{Не очень длинное название для места работы}}
\newcommand{\opponentOneJobPost}       % Оппонент 1, должность
{\todo{старший научный сотрудник}}

\newcommand{\opponentTwoFio}           % Оппонент 2, ФИО
{\todo{Фамилия Имя Отчество}}
\newcommand{\opponentTwoRegalia}       % Оппонент 2, регалии
{\todo{кандидат физико-математических наук}}
\newcommand{\opponentTwoJobPlace}      % Оппонент 2, место работы
{\todo{Основное место работы c длинным длинным длинным длинным названием}}
\newcommand{\opponentTwoJobPost}       % Оппонент 2, должность
{\todo{старший научный сотрудник}}

%% \newcommand{\opponentThreeFio}         % Оппонент 3, ФИО
%% {\todo{Фамилия Имя Отчество}}
%% \newcommand{\opponentThreeRegalia}     % Оппонент 3, регалии
%% {\todo{кандидат физико-математических наук}}
%% \newcommand{\opponentThreeJobPlace}    % Оппонент 3, место работы
%% {\todo{Основное место работы c длинным длинным длинным длинным названием}}
%% \newcommand{\opponentThreeJobPost}     % Оппонент 3, должность
%% {\todo{старший научный сотрудник}}

\newcommand{\leadingOrganizationTitle} % Ведущая организация, дополнительные строки. Удалить, чтобы не отображать в автореферате
{\todo{Омский научно-исследовательский институт приборостроения (АО «ОНИИП»)}}

\newcommand{\defenseDate}              % Защита, дата
{\todo{18 декабря 2021~г.~в~15 часов}}
\newcommand{\defenseCouncilNumber}     % Защита, номер диссертационного совета
{\todo{Д\,212.178.15}}				   %05.13.18 – Математическое моделирование, численные методы и комплексы программ (технические науки).
\newcommand{\defenseCouncilTitle}      % Защита, учреждение диссертационного совета
{\todo{Омском государственном техническом университете}}
\newcommand{\defenseCouncilAddress}    % Защита, адрес учреждение диссертационного совета
{\todo{г.~Омск, проспект~Мира, 11}}
\newcommand{\defenseCouncilPhone}      % Телефон для справок
{\todo{+7~(3812)~65-64-92}}
\newcommand{\defenseCouncilmail}    
 {\todo{E-mail:~dissov\_omgtu@omgtu.ru}}                      %e-mail

\newcommand{\defenseSecretaryFio}      % Секретарь диссертационного совета, ФИО
{\todo{Варепо Лариса Григорьевна}}
\newcommand{\defenseSecretaryRegalia}  % Секретарь диссертационного совета, регалии
{\todo{доктор технических наук, доцент}}            % Для сокращений есть ГОСТы, например: ГОСТ Р 7.0.12-2011 + http://base.garant.ru/179724/#block_30000

\newcommand{\synopsisLibrary}          % Автореферат, название библиотеки
{\todo{Омского государственного технического университета}}
\newcommand{\synopsisDate}             % Автореферат, дата рассылки
{\todo{23 июля 2021 года}}

% To avoid conflict with beamer class use \providecommand
\providecommand{\keywords}%            % Ключевые слова для метаданных PDF диссертации и автореферата
{}
